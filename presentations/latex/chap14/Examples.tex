\section[]{\textgreek{Ασκήσεις επανάληψης}}

% 8
\begin{frame}[t, fragile, shrink]
\frametitle{Πλήθος υπαλλήλων ανά έργο ...}
\begin{minipage}{\wE}
\vspace{-0.5cm}
\begin{exampleblock}{\small Να βρεθεί το πλήθος των εργαζομένων με μισθό
        μικρότερο του 1500, ανά κωδικό και τίτλο έργου,
        με αύξουσα ταξινόμηση ως προς το πλήθος εργαζομένων,
        για έργα που απασχολούν λιγότερο 
        από 5 υπαλλήλους.}
\pause
\en
\begin{SQL}
  SELECT p.proid, p.title, COUNT(*)
    FROM (employees e INNER JOIN workson  w
                         ON e.empid = w.empid)
                      INNER JOIN projects p
                         ON p.proid = w.proid    &pause
   WHERE e.salary < 1500   &pause
GROUP BY p.proid, p.title  &pause
  HAVING COUNT(*) < 5      &pause
ORDER BY COUNT(*) ASC;      
\end{SQL}
\end{exampleblock}
\end{minipage}
\end{frame}


% 13
\begin{frame}[t, fragile, shrink]
\frametitle{Μισθοδοσία ανά τμήμα ...}
\begin{minipage}{\wE}
\vspace{-0.5cm}
\begin{exampleblock}{\small Να βρεθεί ο κωδικός και το όνομα των τμημάτων, καθώς και το άθροισμα μισθοδοσίας
        των υπαλλήλων ανά τμήμα, που απασχολούνται σε όλα τα έργα με πρόοδο πάνω
        από 50\%, και που απασχολούν (τα έργα) περισσότερους από
        έναν υπαλλήλους.}
\pause
\en
\begin{SQL}
  SELECT d.depid, d.depname, SUM(DISTINCT e.salary)
    FROM ((employees e INNER JOIN departments d
                          ON e.depid = d.depid)
                       INNER JOIN workson  w
                          ON e.empid = w.empid)
                       INNER JOIN projects p
                          ON p.proid = w.proid   &pause
   WHERE p.progress > 50          &pause
GROUP BY d.depid, d.depname       &pause
  HAVING COUNT(*) > 1;   
\end{SQL}
\end{exampleblock}
\end{minipage}
\end{frame}


% 14
\begin{frame}[fragile, shrink]
\frametitle{Διευθυντές και έργα}
\begin{minipage}{\wE}
\begin{exampleblock}{\small Να βρεθεί το όνομα του τμήματος,
            το επώνυμο και ο κωδικός του διευθυντή και το πλήθος
            των έργων στα οποία απασχολείται ο κάθε διευθυντής.}
\pause
\en
\begin{SQL}
 SELECT d.depname, e.lastname, w.empid, COUNT(*)
   FROM  (departments d INNER JOIN employees e
                           ON d.manager = e.empid)
                         LEFT JOIN workson w
                           ON e.empid = w.empid  
GROUP BY d.depname, e.lastname, w.empid;
\end{SQL}
\end{exampleblock}
\vspace{0.2cm}
\begin{enumerate}
  \item Σχολιάστε την ύπαρξη της αριστερής σύζευξης.
        Είναι προαιρετική ή απαραίτητη?
  \item Θα έχει νόημα να ήταν αριστερή σύζευξη
        η σύζευξη ανάμεσα σε {\ra departments} και {\ra employees}?
\end{enumerate}

\end{minipage}
\end{frame}


