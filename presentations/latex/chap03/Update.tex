
\section[{\en dbUpdate}]{\textgreek{Ενημέρωση της βάσης δεδομένων}}

%\frame{\el\tableofcontents[currentsection]}

\begin{frame}[t, fragile, shrink]
\frametitle{Ενημέρωση}
\begin{minipage}{\wE}
Εκτός από τις πράξεις επιλογής, η σχεσιακή άλγεβρα έχει ανάγκη από τις πράξεις
ενημέρωσης της βάσης δεδομένων.
Με αυτές υπάρχει η δυνατότητα:
\begin{itemize} \itemsep9pt
    \item {\bf Εισαγωγής} δεδομένων στις σχέσεις, δηλαδή εισαγωγής μιας ή περισσοτέρων
          πλειάδων.
    \item {\bf Τροποποίησης} δεδομένων στις σχέσεις, δηλαδή αλλαγή στις τιμές των
          γνωρισμάτων μιας σχέσης.
    \item {\bf Διαγραφής} δεδομένων από τις σχέσεις, δηλαδή απαλοιφή μιας ή περισσότερων
          πλειάδων της σχέσης.          
\end{itemize}
\end{minipage}  
\end{frame}

\begin{frame}[t, fragile, shrink]
\frametitle{Επιπτώσεις της ενημέρωσης}
\begin{minipage}{\wE}
  \begin{block}{Σχήμα της σχέσης}
    \begin{itemize} \itemsep9pt
      \item Δεν επηρεάζεται το σχήμα της σχέσης.
      \item Δεν μεταβάλλεται ο βαθμός της σχέσης.
    \end{itemize}
  \end{block}
  \begin{block}{Κορμός της σχέσης}
    \begin{itemize} \itemsep9pt
      \item {\bf Εισαγωγή :} αύξηση πληθικότητας.
      \item {\bf Τροποποίηση :} σταθερή πληθικότητα.
      \item {\bf Διαγραφή :} μείωση πληθικότητας.
    \end{itemize}
  \end{block}
\end{minipage}  
\end{frame}

\begin{frame}[t, fragile, shrink]
\frametitle{H σχέση {\en movies}}
\begin{minipage}{\wE}
  \par
  Έστω η σχέση $movies(code, title, year)$: \\
  \par 
  \begin{tabular}{ c l c } \toprule
    {\en\bf code} & {\en\bf title} & {\en\bf year} \\  \midrule
    658 & {\en Blade Runner}  & 1982 \\
    583 & {\en Casablanca}    & 1943 \\
    779 & {\en La Dolce Vita} & 1960 \\
    884 & {\en Paris Texas}   & 1984 \\ \bottomrule
  \end{tabular}
  \par
  \bigskip
  Μια μικρή βάση δεδομένων με τίτλους ταινιών και το έτος πρώτης προβολής. \\
  Το γνώρισμα {\en\em code} είναι πρωτεύον κλειδί.
\end{minipage}  
\end{frame}


\subsection[{\en Insert}]{\textgreek{Εισαγωγή πλειάδων}}

\begin{frame}[t, fragile, shrink]
\frametitle{Εισαγωγή}
\begin{minipage}{\wE}
  \begin{block}{Η εισαγωγή των δεδομένων $E$ (σχεσιακής έκφρασης) στη σχέση $r$, γράφετε ως:}
  \[
     r \leftarrow r \cup E
  \]
  \end{block}
  \begin{exampleblock}{Εισαγωγή της ταινίας {\en Blade Runner} του 1982 με κωδικό 658}
    \en
    \[
      movies \leftarrow movies \cup \{658, \text{'Blade Runner', 1982} \}
    \]
  \end{exampleblock}
\end{minipage}  
\end{frame}



\begin{frame}[t, fragile, shrink]
\frametitle{Παραβίαση πρωτεύοντος κλειδιού}
\begin{minipage}{\wE}

  \el
  \begin{columns}[T]
    \begin{column}{0.5\textwidth}
  \begin{tabular}{ c l c } \toprule
    {\en\bf code} & {\en\bf title} & {\en\bf year} \\  \midrule
    658 & {\en  Blade Runner} & 1982 \\
    583 & {\en Casablanca}    & 1943 \\
    779 & {\en La Dolce Vita} & 1960 \\
    884 & {\en Paris Texas}   & 1984 \\ \bottomrule
        &                     &       \\
  \end{tabular}      
    \end{column}
    \begin{column}{0.5\textwidth}
      {\color{red!90!blue}
      \begin{tabular}{ c l c } \toprule
        {\en\bf code} & {\en\bf title} & {\en\bf year} \\  \midrule
        658 & {\en  Blade Runner} & 1982 \\
        583 & {\en Casablanca}    & 1943 \\
        779 & {\en La Dolce Vita} & 1960 \\
        884 & {\en Paris Texas}   & 1984 \\
            & {\en The Pink Panther} & 1963 \\ \bottomrule            
      \end{tabular}
      }
    \end{column}
  \end{columns}
  \begin{alertblock}{Η τιμή του πρωτεύοντος κλειδιού δεν είναι έγκυρη}
  \en
  \[
    movies \leftarrow movies \cup \{779, \text{'The Pink Panther', 1963} \}
  \]
  \el
  Η παράσταση δεν είναι έγκυρη, η εισαγωγή {\bf πλειάδας} θα αποτύχει.
  \end{alertblock}
\end{minipage}  
\end{frame}


\subsection[{\en Update}]{\textgreek{Τροποποίηση πλειάδων}}

\begin{frame}[t, fragile, shrink]
\frametitle{Τροποποίηση}
\begin{minipage}{\wE}
  \begin{block}{Τελεστής γενικευμένης προβολής}
    \[
       r \leftarrow \Pi_{A_1,A_2,\ldots,A_n}(r)
    \]
  \end{block}
  \begin{exampleblock}{Αλλαγή του έτους κυκλοφορίας}
  \en
  \[
    movies \leftarrow \Pi_{code, title, year=1942} (movies)
  \]
  \end{exampleblock}
\end{minipage}
\end{frame}


\begin{frame}[t, fragile, shrink]
\frametitle{Τροποποίηση του έτους σε 1942}
\begin{minipage}{\wE}
  \begin{block}{Καθολική εφαρμογή της τροποποίησης}
    \[
      movies \leftarrow \Pi_{code, title, year=1942} (movies)
    \]  
    \begin{tabular}{ c l c } \toprule
      {\en\bf code} & {\en\bf title} & {\en\bf year} \\  \midrule
      658 & {\en  Blade Runner} & 1942 \\
      583 & {\en Casablanca}    & 1942 \\
      779 & {\en La Dolce Vita} & 1942 \\
      884 & {\en Paris Texas}   & 1942 \\ \bottomrule
    \end{tabular}
  \end{block}
\end{minipage}
\end{frame}




\begin{frame}[t, fragile, shrink]
\frametitle{Τροποποίηση του έτους σε 1942}
\begin{minipage}{\wE}
  \begin{block}{Επιλεκτική εφαρμογή της τροποποίησης}
    \[
      movies \leftarrow \Pi_{code, title, year=1942}(\sigma_{code=583}(movies))
    \]
    \begin{tabular}{ c l c } \toprule
      {\en\bf code} & {\en\bf title} & {\en\bf year} \\  \midrule
      658 & {\en  Blade Runner} & 1982 \\
      583 & {\en Casablanca}    & 1942 \\
      779 & {\en La Dolce Vita} & 1960 \\
      884 & {\en Paris Texas}   & 1984 \\ \bottomrule
    \end{tabular}
  \end{block}
\end{minipage}
\end{frame}



\subsection[{\en Delete}]{\textgreek{Διαγραφή πλειάδων}}

\begin{frame}[t, fragile, shrink]
\frametitle{Διαγραφή}
\begin{minipage}{\wE}
  \begin{block}{Διαγραφή των δεδομένων που προκύπτουν από μια σχεσιακή παράσταση $E$ στη σχέση $r$}
  \[
    r \leftarrow r - E
  \]
  \end{block}
  \begin{exampleblock}{Διαγραφή της ταινίας {\en Blade Runner} με κωδικό 658}
  \en
  \[
    movies \leftarrow movies - \sigma_{code=658}(movies)
  \]
  \end{exampleblock}
\end{minipage}
\end{frame}





