
\section[{\en $\Pi, \sigma$}]{\textgreek{Οι βασικές πράξεις προβολής και επιλογής}}

\subsection[{\en Project}]{\textgreek{Η σχεσιακή πράξη της προβολής}}

\begin{frame}[t, fragile, shrink]
\frametitle{Προβολή}
\begin{minipage}{\wE}
  \begin{block}{Ορισμός της προβολής}  \Large
    \[ r[X] = \{t[X] \,|\, t \in r \} \]
  \end{block}
  {\bb Προβολή} μιας σχέσης $r(R)$, πάνω στο υποσύνολο γνωρισμάτων της $X$
  ($X \subseteq R$)
  είναι μια σχέση με σχήμα το σύνολο $X$ και κορμό εκείνες τις πλειάδες που
  αντιστοιχούν σε μοναδικές τιμές για τα γνωρίσματα $X$.
  \par Η προβολή συμβολίζεται με το ελληνικό γράμμα $\Pi$:
  \[
    \Pi_{A_1, A_2, \ldots, A_m}(r)
  \]
\end{minipage}
\end{frame}


\begin{frame}[t, fragile, shrink]
\frametitle{Παραδείγματα προβολής}
\begin{minipage}{\wE}
\en
\begin{tabular}{ c c c c c }
  \begin{tabular}{ c c c }  \toprule
    {\bf A} & {\bf B} & {\bf C} \\ \midrule
    5 & a & 30 \\
    2 & b & 10 \\
    3 & c & 20 \\
    5 & b & 10 \\ \bottomrule
  \end{tabular}
&
  \begin{tabular}{ c c }  \toprule
    {\bf A} & {\bf B}  \\ \midrule
    5 & a  \\
    2 & b  \\
    3 & c  \\
    5 & b  \\ \bottomrule
  \end{tabular}
&
  \begin{tabular}{ c c }  \toprule
    {\bf B} & {\bf C}  \\ \midrule
    a & 30 \\
    b & 10 \\
    c & 20 \\         \bottomrule
      &    \\
  \end{tabular}
&
  \begin{tabular}{ c }  \toprule
      {\bf B}   \\ \midrule
      a     \\
      b     \\
      c     \\ \bottomrule
            \\
  \end{tabular}
&
  \begin{tabular}{ c c c }  \toprule
    {\bf A} & {\bf B} & {\bf C} \\ \midrule
    5 & a & 30 \\
    2 & b & 10 \\
    3 & c & 20 \\
    5 & b & 10 \\ \bottomrule
  \end{tabular}
\\
      &                &                &              &         \\
  $r$ & $\Pi_{A,B}(r)$ & $\Pi_{B,C}(r)$ & $\Pi_{B}(r)$ & $\Pi(r)$ \\
\end{tabular}
\el
  \begin{block}{Παρατηρήσεις}
    \begin{itemize}
      \item Απαλοιφή διπλοεγγραφών.
      \item $\Pi(r)$ : Ταυτοτική προβολή.
    \end{itemize}
  \end{block}
\end{minipage}
\end{frame}




\subsection[{\en Project}]{\textgreek{Η σχεσιακή πράξη της επιλογής}}

\begin{frame}[t, fragile, shrink]
\frametitle{Επιλογή}
\begin{minipage}{\wE}
  \begin{block}{Ορισμός της Επιλογής}  \Large
    \[ \sigma_{\phi} (r)  = \{t \in r \,|\, t \; \mathrm{satisfies} \; \phi \} \]
  \normalsize
  Η επιλογή ή αλλιώς και περιορισμός μιας σχέσης $r(R)$,
  είναι μια σχέση που έχει το ίδιο σχήμα $R$
  με τη σχέση $r$
  και κορμό ένα υποσύνολο του κορμού της $r$ που ικανοποιεί μια συνθήκη,
  πχ: $X \; \theta \; Y$.
  \par
  Η επιλογή συμβολίζεται με:
  \[
    \sigma_{X \; \theta \; Y}(r)
  \]
  όπου η συνθήκη περιορισμού είναι μια παράσταση που μπορεί
  να αποτιμηθεί σε \ttrue, \tfalse\ ή \tunk.
  \end{block}
\end{minipage}
\end{frame}


\begin{frame}[t, fragile, shrink]
\frametitle{Διευκρινίσεις για την επιλογή}
\begin{minipage}{\wE}
  \begin{block}{Τελεστές, τελεσταίοι, συγκρίσεις, \tnull}
    \begin{enumerate} \itemsep9pt
      \item Ο τελεστής $\theta$ μπορεί να είναι ένας από $=, \neq, <, \leq, >, \geq$.
      \item Η τιμή ενός γνωρίσματος μπορεί να συγκριθεί με:
        \begin{enumerate}
          \item Την τιμή ενός άλλου γνωρίσματος
          \item Μια κυριολεκτική τιμή
          \item Μια αλγεβρική παράσταση
          \item Μια σχεσιακή παράσταση ({\bb εμφώλευση ερωτημάτων})  
        \end{enumerate}
      \item Οι παραστάσεις μπορούν επίσης να περιέχουν τους λογικούς
            τελεστές  \tand\ ($\wedge$), \tor\ ($\vee$) και \tnot\ ($\neg$). 
      \item Το αποτέλεσμα μιας σύγκρισης μπορεί να είναι \ttrue, \tfalse\ ή \tunk.      
    \end{enumerate}
  \end{block}
\end{minipage}
\end{frame}



\begin{frame}[t, fragile, shrink]
\frametitle{Παραδείγματα επιλογής}
\begin{minipage}{\wE}
  Έστω η σχέση {\en\em employees} : \\ 
  \begin{tabular}{ l l l } \toprule
    {\en\bf empid} & {\en\bf name} & {\en\bf salary} \\ \midrule
    101 & Αθανασίου Μιχ. & 1200 \\
    102 & Βαφειάδης Νικ. & 1150 \\
    104 & Νικολοπούλου Ναν. & 1570 \\
    108 & Βασιλειάδη Μαρ. & 1320 \\ \bottomrule
  \end{tabular}
  \par
  \bigskip
  \par Παραδείγματα: \Large
  \begin{enumerate} \itemsep9pt
    \item $ \sigma_{salary<1300}(employees) $
    \item $ \sigma_{salary\geq1200 \wedge salary\leq1600}(employees) $
    \item $ \sigma_{empid=102}(employees) $
  \end{enumerate}
\end{minipage}
\end{frame}



\begin{frame}[t, fragile, shrink]
\frametitle{Υπάλληλοι με μισθό $<1300$}
\begin{minipage}{\wE}
  \begin{tabular}{ l l l } \toprule
    {\en\bf empid} & {\en\bf name} & {\en\bf salary} \\ \midrule
    101 & Αθανασίου Μιχ. & 1200 \\
    102 & Βαφειάδης Νικ. & 1150 \\
    104 & Νικολοπούλου Ναν. & 1570 \\
    108 & Βασιλειάδη Μαρ. & 1320 \\ \bottomrule
  \end{tabular}
  \begin{block}{Απάντηση}
    \[ \sigma_{salary<1300}(employees) \]
  \end{block}
  \begin{tabular}{ l l l } \toprule
    {\en\bf empid} & {\en\bf name} & {\en\bf salary} \\ \midrule
    101 & Αθανασίου Μιχ. & 1200 \\
    102 & Βαφειάδης Νικ. & 1150 \\ \bottomrule
  \end{tabular}
\end{minipage}
\end{frame}


\begin{frame}[t, fragile, shrink]
\frametitle{Υπάλληλοι με μισθό μεταξύ 1200 και 1600}
\begin{minipage}{\wE}
  \begin{tabular}{ l l l } \toprule
    {\en\bf empid} & {\en\bf name} & {\en\bf salary} \\ \midrule
    101 & Αθανασίου Μιχ. & 1200 \\
    102 & Βαφειάδης Νικ. & 1150 \\
    104 & Νικολοπούλου Ναν. & 1570 \\
    108 & Βασιλειάδη Μαρ. & 1320 \\ \bottomrule
  \end{tabular}
  \begin{block}{Απάντηση}
    \[ \sigma_{salary\geq1200 \wedge salary\leq1600}(employees) \]
  \end{block}
  \begin{tabular}{ l l l } \toprule
    {\en\bf empid} & {\en\bf name} & {\en\bf salary} \\ \midrule
    101 & Αθανασίου Μιχ. & 1200 \\
    104 & Νικολοπούλου Ναν. & 1570 \\
    108 & Βασιλειάδη Μαρ. & 1320 \\ \bottomrule
  \end{tabular}
\end{minipage}
\end{frame}


\begin{frame}[t, fragile, shrink]
\frametitle{Ο υπάλληλος με κωδικό 101}
\begin{minipage}{\wE}
  \begin{tabular}{ l l l } \toprule
    {\en\bf empid} & {\en\bf name} & {\en\bf salary} \\ \midrule
    101 & Αθανασίου Μιχ. & 1200 \\
    102 & Βαφειάδης Νικ. & 1150 \\
    104 & Νικολοπούλου Ναν. & 1570 \\
    108 & Βασιλειάδη Μαρ. & 1320 \\ \bottomrule
  \end{tabular}
  \begin{block}{Απάντηση}
    \[ \sigma_{empid=102}(employees) \]
  \end{block}
  \begin{tabular}{ l l l } \toprule
    {\en\bf empid} & {\en\bf name} & {\en\bf salary} \\ \midrule
    101 & Αθανασίου Μιχ. & 1200 \\ \bottomrule
  \end{tabular}
\end{minipage}
\end{frame}


\subsection[{\en Project}]{\textgreek{Συνδυασμός προβολής και επιλογής}}


\begin{frame}[t, fragile, shrink]
\frametitle{Συνδυασμός προβολής και επιλογής}
\begin{minipage}{\wE}
  \begin{block}{Συνδυασμός σχεσιακών πράξεων}
    \begin{enumerate} \itemsep6pt
      \item Στο αποτέλεσμα μια προβολής μπορεί να εφαρμοστεί επιλογή.
      \item Στο αποτέλεσμα μια επιλογής μπορεί να εφαρμοστεί προβολή.
      \item Στο αποτέλεσμα μια προβολής μπορεί να εφαρμοστεί νέα προβολή.   
      \item Στο αποτέλεσμα μια επιλογής μπορεί να εφαρμοστεί νέα επιλογή.       
    \end{enumerate}
  \end{block}
  \begin{alertblock}{Κλειστότητα}
    Το αποτέλεσμα κάθε σχεσιακής πράξης είναι σχέση.
  \end{alertblock}
\end{minipage}
\end{frame}



\begin{frame}[t, fragile, shrink]
\frametitle{Ο πίνακας {\en\em employees} από τη βάση {\en\em company}}
\begin{minipage}{0.94\textwidth}
  Έστω η σχέση $employees$ με σχήμα: {\color{blue}
  \[
    employees (\underline{empid}, firstname, lastname, depid, salary, hiredate)
  \] }
  \begin{tabular}{ c l l c l l }
    \toprule
{\bf \en empid} & {\bf \en firstname} & {\bf \en lastname} & {\bf \en depid} & {\bf \en salary} & {\bf \en hiredate} \\  
\midrule
102 & Νικηφόρος & Διαμαντίδης & 6 & 1212.50 & 2003-06-02 \\
109 & Μαρία & Αθανασίου & 1 & 2787.69 & 2000-01-26 \\
153 & Μαρία & Αλεβιζάτου & 2 & 1321.92 & 2001-05-15 \\
172 & Χρήστος & Βλάσσης & 3 & 1101.70 & 2000-07-04 \\
189 & Θεόδωρος & Αγγελίνας & 6 & 1908.28 & 2000-06-19 \\
... & ...      & ...       &...&...&... \\
  \end{tabular}
  {\scriptsize 
     \par Δείγμα από τα δεδομένα του πίνακα.  \\ Δείτε τα πλήρη περιεχόμενα εδώ: \\
          {\en \url{http://stavrakoudis.econ.uoi.gr/stavrakoudis/?iid=400}}
  }     
\end{minipage}
\end{frame}

\begin{frame}[t, fragile, shrink]
\frametitle{Ερωτήσεις και απαντήσεις 1--3}
\begin{minipage}{0.94\textwidth}
  \begin{enumerate}
    \item Να βρεθεί το όνομα και το επώνυμο όλων των υπαλλήλων:
          \[ \Pi_{firstname, lastname}(employees) \]
    \item Να βρεθούν οι υπάλληλοι με μισθό μεγαλύτερο του 1500:
          \[ \sigma_{salary>1500}(employees) \]
    \item Να βρεθεί το όνομα και το επώνυμο όλων των υπαλλήλων που παίρνουν μισθό μεγαλύτερο από 1500:
          \[ \Pi_{firstname, lastname} \left(\sigma_{salary>1500}(employees)\right) \]
  \end{enumerate}
\end{minipage}
\end{frame}


\begin{frame}[t, fragile, shrink]
\frametitle{Ερωτήσεις και απαντήσεις 4--6}
\begin{minipage}{0.94\textwidth}
  \begin{enumerate} \setcounter{enumi}{3}
    \item Να βρεθούν οι υπάλληλοι (κωδικός, επώνυμο, τμήμα)
          που δεν εργάζονται στο τμήμα 2 και έχουν μισθό μικρότερο από 1200:
          \[  \Pi_{empid, lastname, depid} \left(\sigma_{(depid\neq 2 \wedge salary < 1200)} (employees) \right) \]
    \item Να βρεθεί το επώνυμο και ο μισθός του υπαλλήλου με κωδικό
          109 μετά την αύξηση 5\% στο μισθό του:
          \[ \Pi_{lastname, salary*1.05} \left(\sigma_{empid=109}(employees)\right) \]
    \item Να βρεθούν οι κωδικοί των υπαλλήλων που δεν εργάζονται στα τμήματα 2, 3, 4:
          \[ \Pi_{empid} \left(\sigma_{ \lnot (depid=2 \vee depid=3 \vee depid=4) } (employees) \right) \]
  \end{enumerate}
\end{minipage}
\end{frame}


