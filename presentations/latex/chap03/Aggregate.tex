
\section[{\en sumFunc}]{\textgreek{Ομαδοποίηση και συνάθροιση}}

\subsection[{\en sumFunc}]{\textgreek{Συναρτήσεις ομαδοποίησης και συνάθροισης}}

\begin{frame}[t, fragile, shrink]
\frametitle{Γενική μορφή συνάθροισης}
Η σύνοψη (ή αλλιώς ομαδοποίηση) ομαδοποιεί πλειάδες μιας σχέσης με βάση κοινές τιμές σε ένα ή περισσότερα γνωρίσματα.
Σε κάθε ένα από τα υποσύνολα των πλειάδων που προκύπτουν μπορεί να εφαρμοστεί μια συναθροιστική συνάρτηση.
\begin{minipage}{\wE}

  \begin{block}{Συνάθροιση:}
    Αν $R$ είναι μια σχεσιακή παράσταση και $X \subset R$ και $A \subset R$ τότε η παράσταση:
      \en
      \[ {}_{X} \calg_{F(A)} (R) \]
      \el
    δηλώνει την ομαδοποίηση ως προς $X$ των πλειάδων της $r(R)$ και την εφαρμογή της συνάρτησης $F$ πάνω στα γνωρίσματα $A$. \\
    Η συνάθροιση δηλώνεται με το καλλιγραφικό {\en G}, $\calg$. \\
    Η χρήση της ομαδοποίησης ($X$) είναι προαιρετική.
  \end{block}
\end{minipage}
\end{frame}


\begin{frame}[t, fragile, shrink]
\frametitle{Συναρτήσεις συνάθροισης}
\begin{minipage}{\wE}
  \begin{block}{5 κύριες συναρτήσεις συνάθροισης:}
    \begin{itemize} \itemsep9pt
      \item \tcount, πλήθος εγγραφών
      \item \tsum,  άθροισμα αριθμητικού γνωρίσματος,
      \item \tavg,  μέση τιμή αριθμητικού γνωρίσματος,
      \item \tmin,  μικρότερη τιμή.
      \item \tmax,  μεγαλύτερη τιμή.
    \end{itemize}
  \end{block}
\end{minipage}
\end{frame}


\begin{frame}[t, fragile, shrink]
\frametitle{Παράδειγμα άθροισης τιμών γνωρίσματος}
\begin{minipage}{\wE}
  Έστω η σχέση {\en\em sales}:
    \begin{tabular}{ l l r } \toprule
        {\en\bf area} & {\en\bf employee} & {\en\bf amount} \\   \midrule
        Ημαθίας  & Ευθυμίου & 1890 \\
        Μαγνησίας & Αλεξανρίδης & 2400 \\
        Καβάλας   & Αλεξανρίδης & 780 \\
        Μαγνησίας & Ευθυμίου & 2100 \\
        Τρικάλων  & Πετρίδης & 4400 \\
        Πιερίας   & Πετρίδης & 1820 \\
        Καβάλας   & Ευθυμίου & 2400 \\        \bottomrule
    \end{tabular}
  \begin{exampleblock}{Να υπολογιστεί το άθροισμα των πωλήσεων}
    \[ \calg_{\mathbf{sum}(amount)} (sales) \]
  \end{exampleblock}
\end{minipage}
\end{frame}


\begin{frame}[t, fragile, shrink]
\frametitle{Ανατομία συνάθροισης με ομαδοποίηση}
\begin{minipage}{\wE}
  { \Large \[ {}_{employee}\calg_{\mathbf{sum}(amount)} (sales) \] }
  \begin{enumerate}
    \item Λαμβάνεται ο κορμός της σχέσης {\en\em sales}, δηλαδή όλες οι πλειάδες.
    \item Ομαδοποίηση ως προς τις τιμές του γνωρίσματος {\en\em employee}.
          εδώ, σε 3 υποσύνολα.
    \item Ο δείκτης {\en\em {sum}(amount)} δεξιά από το $\calg$ (καλλιγραφικό {\en G}),
          σημαίνει πως  θα υπολογιστεί το άθροισμα των τιμών {\en\em amount}
          για κάθε διακριτή τιμή {\en\em employee}.
    \item Το αποτέλεσμα της σχεσιακής παράστασης έχει σχήμα το \{{\en\em employee, sum(amount)}\}.
  \end{enumerate}
\end{minipage}
\end{frame}



\begin{frame}[t, fragile, shrink]
\frametitle{Παράδειγμα άθροισης με ομαδοποίηση}
\begin{minipage}{\wE}
\begin{columns}[T]
  \begin{column}{0.5\textwidth}
  \begin{exampleblock}{Άθροισμα πωλήσεων ανά υπάλληλο}
    \[ {}_{employee}\calg_{\mathbf{sum}(amount)} (sales) \]
        \begin{tabular}{ l r } \toprule
         {\en employee} & {\en sum(amount)} \\  \midrule
         Ευθυμίου & 6390 \\
         Αλεξανρίδης & 3180 \\
         Πετρίδης & 6220 \\     \bottomrule
    \end{tabular}
  \end{exampleblock}    
  \end{column}
  \begin{column}{0.5\textwidth}  \small
    \begin{tabular}{ l l r } \toprule
        {\en\bf area} & {\en\bf employee} & {\en\bf amount} \\   \midrule
        \rowcolor{blue!30!white} Ημαθίας  & Ευθυμίου & 1890 \\
        \rowcolor{green!30!white} Μαγνησίας & Αλεξανρίδης & 2400 \\
        \rowcolor{green!30!white} Καβάλας   & Αλεξανρίδης & 780 \\
        \rowcolor{blue!30!white} Μαγνησίας & Ευθυμίου & 2100 \\
        \rowcolor{red!30!white} Τρικάλων  & Πετρίδης & 4400 \\
        \rowcolor{red!30!white} Πιερίας   & Πετρίδης & 1820 \\
        \rowcolor{blue!30!white} Καβάλας   & Ευθυμίου & 2400 \\        \bottomrule
    \end{tabular}
  \end{column}
\end{columns}
\end{minipage}
\end{frame}



\subsection[{\en groupby}]{\textgreek{Επιπλέον παραδείγματα σύνοψης και συνάθροισης}}

\begin{frame}[t, fragile, shrink]
\frametitle{Ο πίνακας {\en\em employees} από τη βάση {\en\em company}}
\begin{minipage}{0.94\textwidth}
  Έστω η σχέση $employees$ με σχήμα: {\color{blue}
  \[
    employees (\underline{empid}, firstname, lastname, depid, salary, hiredate)
  \] }
  \begin{tabular}{ c l l c l l }
    \toprule
{\bf \en empid} & {\bf \en firstname} & {\bf \en lastname} & {\bf \en depid} & {\bf \en salary} & {\bf \en hiredate} \\  
\midrule
102 & Νικηφόρος & Διαμαντίδης & 6 & 1212.50 & 2003-06-02 \\
109 & Μαρία & Αθανασίου & 1 & 2787.69 & 2000-01-26 \\
153 & Μαρία & Αλεβιζάτου & 2 & 1321.92 & 2001-05-15 \\
172 & Χρήστος & Βλάσσης & 3 & 1101.70 & 2000-07-04 \\
189 & Θεόδωρος & Αγγελίνας & 6 & 1908.28 & 2000-06-19 \\
... & ...      & ...       &...&...&... \\
  \end{tabular}
  {\scriptsize 
     \par Δείγμα από τα δεδομένα του πίνακα.  \\ Δείτε τα πλήρη περιεχόμενα εδώ: \\
          {\en \url{http://stavrakoudis.econ.uoi.gr/stavrakoudis/?iid=400}}
  }     
\end{minipage}
\end{frame}


\begin{frame}[t, fragile, shrink]
\frametitle{Παραδείγματα συναρτήσεων συνάθροισης}
\begin{minipage}{\wE}
  \begin{exampleblock}{Το άθροισμα των μισθών όλων των υπαλλήλων}
    \[ \mathcal{G}_{sum(salary)} (employees) \]
  \end{exampleblock}
    \begin{exampleblock}{Ο μέσος μισθός των υπαλλήλων του τμήματος 3}
    \[ \mathcal{G}_{avg(salary)} ( \sigma_{depid=3} (employees) ) \]
  \end{exampleblock}
  \begin{exampleblock}{Πόσοι υπάλληλοι εργάζονται στο τμήμα 4}
    \[ \mathcal{G}_{count(empid)} ( \sigma_{depid=4} (employees) ) \]
  \end{exampleblock}
  \begin{exampleblock}{Ημερομηνία της πιο πρόσφατης πρόσληψης}
    \[ \mathcal{G}_{max(hiredate)} (employees) \]
  \end{exampleblock}    
\end{minipage}
\end{frame}


\begin{frame}[t, fragile, shrink]
\frametitle{Παραδείγματα συνάθροισης με ομαδοποίηση}
\begin{minipage}{\wE}
  \begin{exampleblock}{Ο μικρότερος μισθός ανά τμήμα υπαλλήλων}
    \[ {}_{depid} \mathcal{G}_{min(salary)} (employees) \]
  \end{exampleblock}
  \begin{exampleblock}{Να βρεθεί το εύρος μισθού ανά τμήμα}
    \[ {}_{depid} \mathcal{G}_{max(salary)-min(salary)}  (employees) \]
  \end{exampleblock}
  \begin{exampleblock}{Πλήθος υπαλλήλων ανά τμήμα με μισθό $>1300$}
    \[ {}_{depid} \mathcal{G}_{count(depid)}
       \left(
         \sigma_{salary>1300} (employees)
       \right) 
    \]
  \end{exampleblock}    
\end{minipage}
\end{frame}


\begin{frame}[t, fragile, shrink]
\frametitle{Περιορισμός μετά από ομαδοποίηση}
\begin{minipage}{\wE}
  \begin{exampleblock}{Τμήματα με περισσότερους από 4 υπαλλήλους}
    \[  
      \sigma _{count(depid) >4} 
      \left(
        {}_{depid}  \calg_{count(empid)}  (employees)  
      \right)
   \]
  \end{exampleblock}
  \begin{columns}[T]
    \begin{column}{0.4\textwidth}
      \begin{tabular} { c c }
        \begin{tabular}{r r} \toprule
          {\en\bf depid} & {\en \bf count(empid)} \\ \midrule
          1    &     3 \\
          2    &     4 \\
          \rowcolor{blue!30!white}
          3    &     9 \\
          \rowcolor{blue!30!white}
          4    &     5 \\
          \rowcolor{blue!00!white}
          5    &     2 \\
          \rowcolor{blue!30!white}
          6    &     7 \\ \bottomrule
\end{tabular} 
      \end{tabular}
    \end{column}
    \begin{column}{0.5\textwidth}
      Μόνο οι σκιασμένες γραμμές στο αποτέλεσμα, δηλαδή μόνο εκείνες για τις οποίες 
      η σύγκριση 
      \[ count(depid) > 4 \]
      αποδίδει την τιμή {\sq \ttrue}.
    \end{column}
  \end{columns} 
\end{minipage}
\end{frame}


