
\section[\textgreek{Κλειστότητα}]{\textgreek{Κλειστότητα και συμβατότητα τύπου}}


\begin{frame}[t, fragile, shrink]
\frametitle{Σχεσιακή άλγεβρα}
\begin{minipage}{\wE}
  \large
  \begin{itemize}{\itemsep9pt}
    \item Η {\crr σχεσιακή άλγεβρα} είναι μια διαδικαστική ({\en procedural}) γλώσσα.
    \item Διαθέτει ένα σύνολο τελεστών για σχεσιακές πράξεις.
    \item Βασικές πράξεις: Προβολή, Επιλογή, Ένωση, Διαφορά, Καρτεσιανό Γινόμενο.
    \item Παράγωγες πράξεις: Σύζευξη, Διαίρεση, Τομή.
    \item Επιπλέον πράξεις: Συνάθροιση, Μετονομασία, Εισαγωγή, Διαγραφή, Ενημέρωση.
  \end{itemize}
\end{minipage}
\end{frame}

\begin{frame}[t, fragile, shrink]
\frametitle{Κλειστότητα}
\begin{minipage}{\wE}
  \large
  \begin{itemize}{\itemsep6pt}
    \item Η {\crr σχεσιακή άλγεβρα} και ο σχεσιακός λογισμός παρέχουν
          ένα σύνολο από τελεστές για πράξεις ανάμεσα σε σχέσεις.
    \item Οι πράξεις με σχέσεις παράγουν νέες σχέσεις.
    \item Το αποτέλεσμα της πράξης έχει καθορισμένο {\cbb βαθμό} και {\cbb πληθικότητα}.
  \end{itemize}
  \begin{block}{Κλειστότητα}
    Tο αποτέλεσμα οποιασδήποτε σχεσιακής πράξης είναι {\crr σχέση}.
  \end{block}
\end{minipage}
\end{frame}




\begin{frame}[t, fragile, shrink]
\frametitle{Συμβατότητα τύπου}
\begin{minipage}{\wE}
\begin{block}{Ορισμός}
Δύο σχέσεις $r$ και $s$, έχουν \textbf{συμβατότητα τύπου}, αν και μόνο αν:
\begin{itemize}
  \item Έχουν τον ίδιο βαθμό, δηλαδή έχουν το ίδιο πλήθος γνωρισμάτων.
  \item Τα αντίστοιχα γνωρίσματα έχουν το ίδιο πεδίο ορισμού.
\end{itemize}
\end{block}
\pause
\begin{exampleblock}{Παράδειγμα}
\begin{figure}[btp]
  \centering
  \en
  \textbf{r} \vspace{0.5cm}
    \begin{tabular}{ c c c } \toprule
        {\bf A} & {\bf B} & {\bf C} \\ \midrule
        1 & b & 10 \\
        5 & a & 30 \\
        3 & c & 20 \\   \bottomrule
    \end{tabular} \phantom{nocnocnoc}
      \textbf{s}  \vspace{0.5cm}
    \begin{tabular}{ c c c } \toprule
        {\bf A} & {\bf B} & {\bf C} \\ \midrule
        5 & a & 20 \\
        2 & b & 10 \\
        3 & c & 20 \\  \bottomrule
    \end{tabular}
\end{figure}
\end{exampleblock}
\end{minipage}
\end{frame}


\begin{frame}[t, fragile, shrink]
\frametitle{Παραδείγματα μη συμβατότητας τύπου}
\begin{minipage}{\wE}
  \begin{exampleblock}{Παράδειγμα}
      \en
      \textbf{r} \vspace{0.5cm}
      \begin{tabular}{ c c c } \toprule
        {\bf A} & {\bf B} & {\bf C} \\ \midrule
        1 & b & 10 \\
        5 & a & 30 \\
        3 & c & 20 \\   \hline
      \end{tabular} \phantom{}
      \textbf{s}  \vspace{0.5cm}
      \begin{tabular}{ c c c } \toprule
        {\bf A} & {\bf B} & {\bf C} \\ \midrule
        5 & a & 20 \\
        2 & b & 10 \\
        3 & c & 20 \\   \hline
      \end{tabular} \phantom{}
      \textbf{t}
      \begin{tabular}{ c c } \toprule
        {\bf A} & {\bf B}  \\ \midrule
        5 & b \\
        2 & b \\
        3 & c \\    \hline
      \end{tabular} \phantom{}
      \textbf{u}
      \begin{tabular}{ c c c } \toprule
        {\bf A} & {\bf B} & {\bf C} \\ \midrule
        5 & b & a \\
        2 & b & b \\
        3 & c & b \\    \hline
      \end{tabular}
    \el

  \end{exampleblock}
    \begin{enumerate}
      \item Οι σχέσεις $r$ και $t$ δεν έχουν συμβατότητα τύπου.
      \item Οι σχέσεις $r$ και $u$ δεν έχουν συμβατότητα τύπου.
      \item Οι σχέσεις $s$ και $t$ δεν έχουν συμβατότητα τύπου.
      \item Οι σχέσεις $s$ και $u$ δεν έχουν συμβατότητα τύπου.
      \item Οι σχέσεις $t$ και $u$ δεν έχουν συμβατότητα τύπου.
    \end{enumerate}
\end{minipage}
\end{frame}