
\begin{frame}[shrink]
\frametitle{Η σχεσιακή πράξη της σύζευξης}
Η σύζευξη μοιάζει με το γινόμενο, ωστόσο έχει μια σημαντική διαφορά:

Οι πλειάδες του αποτελέσματος
δεν είναι όλοι οι πιθανοί συνδυασμοί, αλλά μόνο οι συνδυασμοί που ικανοποιούν κάποια συνθήκη.

Υπάρχουν διάφορες παραλλαγές συζεύξεων, με σημαντικότερη και περισσότερο διαδεδομένη τη
 φυσική σύζευξη, η οποία στηρίζεται στην παρουσία γνωρισμάτων
με κοινές τιμές στις δύο σχέσεις.
\end{frame}

\begin{frame}[shrink]
\frametitle{Ορισμός της σύζευξης}
\begin{block}{Ορισμός}
Αν η $r$ είναι σχέση με σχήμα $R=\{X,Y\}$ και $s$ είναι σχέση με σχήμα $S=\{Y,Z\}$, τότε η φυσική
σύζευξη των $R$ και $S$ είναι μια σχέση με σχήμα $\{X,Y,Z\}$ και κορμό το σύνολο των συνδυασμών των
πλειάδων της $r$ και $s$ για τις οποίες οι τιμές στο κοινό γνώρισμα $Y$ ταυτίζονται.

Δηλαδή, μια πλειάδα της $r$, θα συνδυαστεί με μια πλειάδα της $s$,
αν και μόνο αν, οι τιμές στο κοινό γνώρισμα $Y$ ταυτίζονται μεταξύ τους.

Η φυσική σύζευξη των σχέσεων $r$ και $s$ συμβολίζεται με $s \bowtie s$,
ή $r \quad NATURAL \quad JOIN \quad s$, ή απλά $r \quad JOIN \quad s$.
\end{block}
\end{frame}



\begin{frame}
\frametitle{Ιδιότητες της σύζευξης}
\begin{block}{Αντιμεταθετική ιδιότητα}
\[
  r \bowtie s \neq s \bowtie r
\]
\[ \Pi_L (r \bowtie s) = \Pi_L (s \bowtie r) \]
\end{block}
\begin{block}{Προσεταιριστική ιδιότητα}
\[
r \bowtie (s \bowtie t) = (r \bowtie s) \bowtie  t
\]
\end{block}
\end{frame}


\begin{frame}[fragile]
\frametitle{Φυσική σύζευξη με {\sq SQL}}
\begin{block}{Παράδειγμα σύζευξης} \en
\begin{lstlisting}
  SELECT *
    FROM departments NATURAL JOIN employees
\end{lstlisting}
\el
Η φυσική σύζευξη στην {\sq SQL} γίνεται είτε με τη φράση {\sq NATURAL JOIN}
\end{block}
\end{frame}

\begin{frame}[fragile]
\frametitle{Παράδειγμα φυσικής σύζευξης}

{\bb \en departments}:
\begin{tabular}{|c l|} \hline
{\en depid} & {\en depname} \\ \hline
1 & Μελετών \\
2 & Λογιστήριο \\
3 & Διαφήμισης \\ \hline
\end{tabular}

{\bb \en employees}:
\begin{tabular}{|c l c|} \hline
{\en empid} & {\en empname} & {\en depid} \\ \hline
102 & Αποστολάκης & 2 \\
154 & Βασιλάκης   & 1 \\
132 & Χρηστάκης   & 2\\
432 & Δημητράκης  & 3 \\
203 & Κωστάκης    & 1 \\ \hline
\end{tabular}

{\bb \en departments $\bowtie$ employees}: \\
\begin{tabular}{|c l c l|} \hline
{\en depid} & {\en depname} & {\en empid} & {\en empname} \\ \hline
1 & Μελετών & 154 & Βασιλάκης  \\
1 & Μελετών & 203 & Κωστάκης   \\
2 & Λογιστήριο & 102 & Αποστολάκης \\
2 & Λογιστήριο & 132 & Χρηστάκης \\
3 & Διαφήμισης & 432 & Δημητράκης \\ \hline
\end{tabular}

\end{frame}



\begin{frame}
\frametitle{Η σχεσιακή πράξη της σύζευξης θ}
\begin{block}{Ορισμός}
Αν η $r$ είναι σχέση με σχήμα $R=\{A_1, A_2, \ldots, A_n\}$,

$s$ είναι σχέση
με σχήμα $S=\{B_1, B_2, \ldots, B_m\}$,

τα γνωρίσματα $A_i$ και $B_j$ έχουν το ίδιο πεδίο ορισμού,

και $\theta$ είναι τελεστής σύγκρισης, $\theta \in \{=,\neq,<,\leq,>,\geq\}$,

τότε η $\theta$ σύζευξη των $R$ και $S$, $R\bowtie_{A_i \theta B_j}S$, είναι μια σχέση με σχήμα
το σύνολο των γνωρισμάτων των $R$ και $S$, $\{A_1, A_2, \ldots, A_n\, B_1, B_2, \ldots, B_m\}$

και κορμό το σύνολο των πλειάδων από κάθε συνδυασμό των πλειάδων των $R$ και $S$,
που ικανοποιούν τη συνθήκη $A_i \theta B_j$.
\end{block}
Ικανοποιούν τη συνθήκη:{\sq TRUE} όχι {\sq FALSE} ή {\sq NULL}
\end{frame}



\begin{frame}
\frametitle{Παρατηρήσεις για τη θ σύζευξη}
\begin{itemize}
  \item Το αποτέλεσμα είναι μια σχέση με βαθμό $m+n$,
        και πληθικότητα ανάμεσα στο 0 και στο $n_R \cdot n_S$
  \item Αν κάποια πλειάδα έχει στο γνώρισμα ($A_i$ ή $B_j$ ) που συμμετέχει
στη σύζευξη {\sq NULL} τιμή, τότε δεν συμμετέχει στο αποτέλεσμα.
Με αυτή την έννοια το αποτέλεσμα της σύζευξης,
ενδεχομένως, περιέχει λιγότερη πληροφορία, από ότι οι επιμέρους σχέσεις.
  \item Αν ο τελεστής $\theta$ είναι το $=$ τότε η σύζευξη καλείται ισοσύζευξη,
        και πρακτικά είναι φυσική σύζευξη.
  \item Η σύζευξη $\theta$ ($\theta JOIN$) είναι γενίκευση της σύζευξης,
       και στη ουσία πρόκειται για συνδυασμό των πράξεων γινομένου και επιλογής,
       έτσι μπορεί να γραφεί: \\
       \[ (r \times s) \quad WHERE \quad X \quad \theta \quad Y \]
       όπου ο τελεστής $\theta$ δεν είναι κατά ανάγκη ισότητα.
\end{itemize}
\end{frame}



\begin{frame}[fragile]
\frametitle{θ σύζευξη με {\sq SQL}}
\begin{block}{Παράδειγμα σύζευξης} \en
\begin{lstlisting}
  SELECT *
    FROM deprtments, employees
   WHERE departments.code = employees.depid
\end{lstlisting}
\el
Η σύζευξη θ στην {\sq SQL} γίνεται είτε με τη φράση {\sq WHERE}
\end{block}
\end{frame}

\begin{frame}[fragile]
\frametitle{Παράδειγμα  σύζευξης θ }

{\bb \en departments}:
\begin{tabular}{|c l|} \hline
{\en depcode} & {\en depname} \\ \hline
1 & Μελετών \\
2 & Λογιστήριο \\
3 & Διαφήμισης \\ \hline
\end{tabular}

{\bb \en employees}:
\begin{tabular}{|c l c|} \hline
{\en empid} & {\en empname} & {\en depid} \\ \hline
102 & Αποστολάκης & 2 \\
154 & Βασιλάκης   & 1 \\
132 & Χρηστάκης   & 2\\
432 & Δημητράκης  & 3 \\
203 & Κωστάκης    & 1 \\ \hline
\end{tabular}

{\bb \en departments $\times$ employees WHERE depcode=depid}: \\
\begin{tabular}{|c l c l c|} \hline
{\en depcode} & {\en depname} & {\en empid} & {\en empname} & {\en depid} \\ \hline
1 & Μελετών & 154 & Βασιλάκης  & 1\\
1 & Μελετών & 203 & Κωστάκης   & 1\\
2 & Λογιστήριο & 102 & Αποστολάκης & 2\\
2 & Λογιστήριο & 132 & Χρηστάκης  & 2\\
3 & Διαφήμισης & 432 & Δημητράκης & 3\\ \hline
\end{tabular}

\end{frame}



\begin{frame}[shrink]
\frametitle{Ορισμός εξωτερική σύζευξης}
\begin{block}{}
Αν η $r$ είναι σχέση με σχήμα
$R= \{ A_1, A_2, \ldots, A_n, B_1, B_2, \ldots, B_k \} $,

$s$ είναι σχέση
με σχήμα $S=\{B_1, B_2, \ldots, B_k, C_1, C_2, \ldots, C_m\}$,

τότε η εξωτερική σύζευξη
$t = r \ojoin s$ έχει σχήμα $T=\{A_1, A_2, \ldots, A_n, B_1, B_2, \ldots, B_k, C_1, C_2, \ldots, C_m\}$ και κορμό που
αποτελείται από :
\begin{enumerate}
  \item τις πλειάδες της εσωτερικής σύζευξης  των $r \bowtie s$
  \item τις πλειάδες της σχέσης $r$ που δεν έχουν ταιριαστές τιμές στην $s$,
        με τιμές {\en NULL} στα αντίστοιχα γνωρίσματα της $s$
  \item τις πλειάδες της σχέσης $s$ που δεν έχουν ταιριαστές τιμές στην $r$,
        με τιμές {\en NULL} στα αντίστοιχα γνωρίσματα της $r$
\end{enumerate}
\end{block}



\begin{frame}[shrink]
\frametitle{Εξωτερική σύζευξη}
Η εξωτερική σύζευξη είναι επέκταση της σύζευξης, στην περίπτωση που υπάρχουν πλειάδες σε μία
ή περισσότερες σχέσεις, χωρίς ταιριαστές τιμές.

Για παράδειγμα θεωρείστε τις δύο σχέσεις του σχήματος,
που παριστάνουν ένα δείγμα από τα υποκαταστήματα (Υ) και τους πελάτες (Π)
μιας εταιρείας.

Θέλουμε να βρούμε το αποτέλεσμα της εξωτερικής σύζευξης
των δύο σχέσεων με βάση την πόλη:
\[
    \text{Υποκαταστήματα} \quad \ojoin \quad \text{Πελάτες}
\]
\end{frame}





\begin{frame}[shrink]
\frametitle{Παράδειγμα εξωτερική σύζευξης}

\begin{columns}
\begin{column}{0.5\textwidth}
{\bb Υποκαταστήματα}\\
\begin{tabular}{|c|c|} \hline
          {Κωδικός} & {Πόλη} \\ \hline  
          1 & Αθήνα \\
          2 & Πάτρα \\
          3 & Θεσσαλονίκη \\ \hline
\end{tabular}
\end{column}

\begin{column}{0.5\textwidth}
{\bb Πελάτες}\\
\begin{tabular}{|c|c|} \hline
          { Όνομα} & {Πόλη} \\ \hline  
          Νίκος & Πάτρα \\
          Βάσω  & Κοζάνη \\
          Αγγελική & Πάτρα \\
          Βασίλης & Αθήνα \\ \hline
\end{tabular}
\end{column}
\end{columns}

 \bigskip
 {\bb Υποκαταστήματα $\ojoin$ Πελάτες} \\
 \begin{tabular}{|c|c|c|} \hline
       {Κωδικός} & {Πόλη} & {Όνομα}  \\ \hline  
       1 & Αθήνα & Βασίλης  \\
       2 & Πάτρα & Νίκος  \\
       2 & Πάτρα & Αγγελική \\
       3 & Θεσσαλονίκη & {\en NULL}   \\
        {\en NULL} & {\en NULL} & Βάσω  \\ \hline
 \end{tabular}
\end{frame}



\begin{frame}[shrink]
\frametitle{Παράδειγμα δεξιάς εξωτερική σύζευξης}
 
\begin{columns}
\begin{column}{0.5\textwidth}
{\bb Υποκαταστήματα}\\
\begin{tabular}{|c|c|} \hline
          {Κωδικός} & {Πόλη} \\ \hline \hline
          1 & Αθήνα \\
          2 & Πάτρα \\
          3 & Θεσσαλονίκη \\ \hline
\end{tabular}
\end{column}

\begin{column}{0.5\textwidth}
{\bb Πελάτες}\\
\begin{tabular}{|c|c|} \hline
          { Όνομα} & {Πόλη} \\ \hline \hline
          Νίκος & Πάτρα \\
          Βάσω  & Κοζάνη \\
          Αγγελική & Πάτρα \\
          Βασίλης & Αθήνα \\ \hline
\end{tabular}
\end{column}
\end{columns}

\bigskip
{\bb Υποκαταστήματα $\rojoin$ Πελάτες} \\
  \begin{tabular}{|c|c|c|} \hline
        {Κωδικός} & {Πόλη} & {Όνομα}  \\ \hline 
        1 & Αθήνα & Βασίλης  \\
        2 & Πάτρα & Νίκος  \\
        2 & Πάτρα & Αγγελική \\
        {\en NULL} &  Βάσω & Κοζάνη   \\ \hline
  \end{tabular}
\end{frame}


\begin{frame}[shrink]
\frametitle{Παράδειγμα αριστερής εξωτερική σύζευξης}
 
\begin{columns}
\begin{column}{0.5\textwidth}
{\bb Υποκαταστήματα}\\
\begin{tabular}{|c|c|} \hline
          {Κωδικός} & {Πόλη} \\ \hline 
          1 & Αθήνα \\
          2 & Πάτρα \\
          3 & Θεσσαλονίκη \\ \hline
\end{tabular}
\end{column}

\begin{column}{0.5\textwidth}
{\bb Πελάτες}\\
\begin{tabular}{|c|c|} \hline
          { Όνομα} & {Πόλη} \\ \hline 
          Νίκος & Πάτρα \\
          Βάσω  & Κοζάνη \\
          Αγγελική & Πάτρα \\
          Βασίλης & Αθήνα \\ \hline
\end{tabular}
\end{column}

\end{columns}

 \bigskip
 { Υποκαταστήματα $\lojoin$ Πελάτες} \\
 \begin{tabular}{|c|c|c|} \hline
       {Κωδικός} & {Πόλη} & {Όνομα}  \\ \hline  
      1 & Αθήνα & Βασίλης  \\  
      2 & Πάτρα & Νίκος   \\  
      2 & Πάτρα & Αγγελική  \\  
      3 & Θεσσαλονίκη & {\en NULL}     \\ \hline
 \end{tabular}
\end{frame}



% part III




\begin{frame}[shrink]
\frametitle{Η σχεσιακή πράξη της διαίρεσης}

Αν η $r$ είναι σχέση με σχήμα $R=\{X,Y\}$ και $s$ είναι μια σχέση
με σχήμα $S=\{Y\}$, δηλαδή $S \subseteq R$, τότε,
το αποτέλεσμα της διαίρεσης $t = r \div s$, 
είναι μια  σχέση $t$ με σχήμα $\{X\}$. 
Δηλαδή, η $t$ είναι μια σχέση με σχήμα τη διαφορά
$R-S=\{X\}$, δηλαδή εκείνα τα γνωρίσματα της $r$ που δεν ανήκουν στην $s$.

Ο κορμός της $t$ αποτελείται από εκείνες τις πλειάδες
της σχέσης $r$ για οποίες τα (κοινά) γνωρίσματα $\{X\}$ των σχέσεων $r$ και $s$ 
έχουν τιμές που ταυτίζονται. 
Μια πλειάδα $t$, με γνωρίσματα $R-S=\{X\}$, ανήκει στο αποτέλεσμα όταν:

\begin{enumerate}
  \item $t \in \Pi_{R-S}(r)$ 
  \item  $ \forall \; t_s \in s, \quad   \exists \; t_r \in r$ έτσι ώστε:
  \begin{itemize}
    \item $t_r[S] = t_s[S]$
    \item $t_r[R-S] = t$ 	
  \end{itemize}
\end{enumerate}

\end{frame}



\begin{frame}[shrink]
\frametitle{Διαίρεση, διαφορά και καρτεσιανό γινόμενο}
\begin{block}{Η διαίρεση είναι παράγωγη πράξη}
\[
 r \div s = \Pi_{R-S}(r) - \Pi_{R-S}\left( (\Pi_{R-S}(r) \times s  ) - \Pi_{R-S,S}(r) \right)
\]
\end{block}

\begin{block}{Η διαίρεση είναι το αντίστροφο του γινομένου στην άλγεβρα των αριθμών, και το ίδιο μπορεί
να ειπωθεί για τη σχεσιακή άλγεβρα}
Έτσι, αν $t = r \div s$, τότε η σχέση $t \times s$, έχει
συμβατότητα τύπου με την $r$
\end{block}

\end{frame}



\begin{frame}[shrink]
\frametitle{Παράδειγμα διαίρεσης}

\begin{columns}

\begin{column}{0.4\textwidth} Αθλούνται
  \begin{tabular}{ l l }           \hline
  \emph{Όνομα} & \emph{Άθλημα} \\  \hline
  Νίκος & Ποδόσφαιρο \\ 
  Ανδρέας & Κολύμπι \\ 
  Βασίλης & Στίβος \\ 
  Ανδρέας & Ποδόσφαιρο \\ 
  Χρήστος & Κολύμπι  \\
  Βασίλης & Ποδόσφαιρο \\
  Νίκος   & Στίβος  \\
  Γιώργος & Ποδόσφαιρο \\
  Χρήστος & Στίβος     \\           \hline
  \end{tabular}
\end{column}
\hspace{1cm}

\begin{column}{0.3\textwidth} Πρωτάθλημα
  \begin{tabular}{ c } \hline
  \emph{Άθλημα }\\ \hline  
  Ποδόσφαιρο \\
  Στίβος \\ \hline \\ \\ \\
  \end{tabular}
\end{column}

\begin{column}{0.3\textwidth} Συμμετοχή
  \begin{tabular}{ c } \hline
  \emph{Όνομα} \\ \hline 
  Νίκος \\
  Βασίλης \\    \hline   
  \end{tabular}
\end{column}
\end{columns}

\[ \text{Συμμετοχή} = \text{Αθλούνται} \div \text{Πρωτάθλημα} \]

\begin{itemize}%{$\hookrightarrow$}
\item 
Διαιρετέος είναι η σχέση Αθλούνται (α) και διαιρέτης η σχέση \emph{Πρωτάθλημα} (π).
          Το αποτέλεσμα είναι η σχέση Συμμετοχή (σ)
\item 
Το αποτέλεσμα έχει σχήμα τη διαφορά: $\Sigma=A-\Pi$.
          Η σχέση Συμμετοχή έχει εκείνα τα γνωρίσματα της σχέσης \emph{Αθλούνται}, που
          δεν υπάρχουν στη σχέση \emph{Πρωτάθλημα}
\item 
Για το κορμό της σχέσης {\em Συμμετοχή}, πρέπει να βρεθούν εκείνες οι πλειάδες της 
σχέσης {\em Αθλούνται}, για τις οποίες το {\em Άθλημα}
υπάρχει στις πλειάδες που αντιστοιχούν σε όλες τις πλειάδες της σχέση {\em Πρωτάθλημα}.
Δηλαδή, θα ληφθούν τα ονόματα των μαθητών που αθλούνται σε όλα τα αθλήματα
τα οποία υπάρχουν στη λίστα αθλημάτων για το πρωτάθλημα.
\end{itemize}
\end{frame}

