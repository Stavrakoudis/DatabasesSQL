\el
\section[\textgreek{Περιγραφή}] {\textgreek {Κεντρικές έννοιες του σχεσιακού μοντέλου} }

\subsection[\textgreek{Περί}]{\textgreek{Ορισμοί για τις σχέσεις}}

\begin{frame}[t, fragile, shrink]
\frametitle{Τι είναι σχέση?}
\begin{minipage}{\wE}
  \begin{tabular}{ c l c } \toprule
    {\bf Κωδικός} & {\bf Όνομα} & {\bf Εξάμηνο} \\ \midrule 
    504 & Βάσεις Δεδομένων & 5 \\ 
    404 & Μακροοικονομική Θεωρία ΙΙ & 4 \\  
    303 & Προγραμματισμός Υπολογιστών Ι & 3 \\   
    604 & Πληροφοριακά Συστήματα Διοίκησης & 6 \\  \bottomrule
  \end{tabular}
  \bigskip \par Η πιο απλή πρακτική αναπαράσταση μιας σχέσης, είναι ένας πίνακας
  δεδομένων δύο διαστάσεων.
  Το παραπάνω σχήμα  απεικονίζει ένα παράδειγμα μιας σχέσης: της σχέσης {\bb μάθημα} από
  το πρόγραμμα σπουδών ενός τμήματος πανεπιστημίου. 
\end{minipage}
\end{frame}


\begin{frame}[t, fragile, shrink]
\frametitle{Αντιστοιχία πίνακα με σχέση}
\begin{minipage}{\wE}
  \pause
  \begin{enumerate}[<+->] \itemsep 6pt
    \item Η αντιστοιχία είναι άτυπη, μια σχέση δεν είναι ακριβώς ένας πίνακας.
    \item Η σχέση έχει μια {\crr επικεφαλίδα}, την πρώτη
          γραμμή του πίνακα, που συνιστά το {\crr σχήμα της σχέσης}.
    \item Το σχήμα της σχέσης είναι ένα
          σύνολο από γνωρίσματα, πχ  \{Κωδικός, Όνομα, Εξάμηνο\}.
    \item Το σύνολο \{504, Βάσεις Δεδομένων, 5\} 
          είναι μια πλειάδα (ή συστοιχία) της σχέσης {\crr Μαθήματα}.
    \item Μια σχέση έχει ακριβώς ένα καθορισμένο σχήμα, 
          έχει όμως, ενδεχομένως, πολλές πλειάδες.
    \item Οι τιμές κάθε γνωρίσματος προέρχονται από το κάποιο {\crr πεδίο ορισμού}.
\end{enumerate}
\end{minipage}
\end{frame}


\begin{frame}[t, fragile, shrink]
\frametitle{Διευκρινίσεις για τις σχέσεις}
\begin{minipage}{\wE}
  \pause
  \begin{enumerate}[<+->] \itemsep 3pt
    \item Μια σχεσιακή βάση δεδομένων καταγράφει δεδομένα μέσα σε {\crr σχέσεις}, 
          και μόνο σε αυτές.
    \item Αντικείμενα και γεγονότα γίνονται αντιληπτά στη βάση δεδομένων,
          ως τιμές που αντιστοιχούν στα {\crr γνωρίσματα} μιας σχέσης.
    \item Η σχέση είναι ένα {\crr σύνολο από γνωρίσματα}, το καθένα με διαφορετικό όνομα,
          και κάποιο πεδίο ορισμού.
    \item Η πλειάδα είναι ένα {\crr σύνολο από τιμές} που προέρχονται από το πεδίο τιμών του κάθε γνωρίσματος.
    \item Μια σχέση έχει ένα καθορισμένο σύνολο γνωρισμάτων, 
          το οποίο γενικά μένει σταθερό ως προς το χρόνο χρήσης
          της βάσης δεδομένων.
    \item Το σύνολο αυτό λέγεται επικεφαλίδα της σχέσης, ή {\crr σχήμα της σχέσης}. 
  \end{enumerate}
\end{minipage}
\end{frame}


\begin{frame}[t, fragile, shrink]
\frametitle{Ενημέρωση σχέσεων}
\begin{minipage}{\wE}
  \pause
  \begin{enumerate}[<+->] \itemsep 6pt
    \item Με τον όρο {\crr ενημέρωση} της  βάσης δεδομένων εννοείται η ενημέρωση μιας
         (ή και περισσότερων) σχέσης (ή σχέσεων) της βάσης δεδομένων.
    \item Η ενημέρωση μιας σχέσης γίνεται με την έννοια της {\crr πλειάδας},
          ενός συνόλου τιμών που αντιστοιχούν στα γνωρίσματα της σχέσης.
    \item Η ενημέρωση γίνεται με τρεις πράξεις:
          \begin{itemize}
            \item Εισαγωγή πλειάδων
            \item Διαγραφή πλειάδων
            \item Τροποποίηση πλειάδων
          \end{itemize}
  \end{enumerate}
\end{minipage}
\end{frame}



\begin{frame}[t, fragile, shrink]
\frametitle{Ενημέρωση σχέσεων : Τροποποίηση}
  \vspace{-0.2cm}
  \begin{tabular}{ c p{7cm} c } \toprule
    {\bf Κωδικός} & {\bf Όνομα} & {\bf Εξάμηνο} \\ \midrule 
    504 & Βάσεις Δεδομένων & 5 \\ 
    404 & Μακροοικονομική Θεωρία ΙΙ & 4 \\  
    303 & \sout{Προγραμματισμός Υπολογιστών Ι}  \newline{} Εισαγωγή στον Προγραμματισμό & 3 \\   
    604 & Πληροφοριακά Συστήματα Διοίκησης & 6 \\  \bottomrule
  \end{tabular}
  \vspace{-0.5cm}
  \begin{minipage}{\wE}
    \begin{exampleblock}{Παράδειγμα} \small
      Αν το όνομα του μαθήματος {\cbb ((Προγραμματισμός Υπολογιστών Ι))} αλλάξει 
      σε {\crr ((Εισαγωγή στον Προγραμματισμό))}
      τότε αυτό που τροποποιήθηκε είναι η
      {\crr πλειάδα} με κωδικό 303. \\
      Άλλαξε δηλαδή τιμές κάποιο σύνολο καθώς η
      μεταβολή μιας τιμής μεταβάλει όλο το σύνολο τιμών, η {\crr ενημέρωση} των σχέσεων
      γίνεται {\crr κατά πλειάδες}.
    \end{exampleblock}
  \end{minipage}
\end{frame}



\begin{frame}[t, fragile, shrink]
\frametitle{Σχήμα σχέσης}
\begin{minipage}{\wE}
  \begin{block}{Σχήμα σχέσης}
    Σχήμα μιας σχέσης είναι το σύνολο των γνωρισμάτων της.
      \[ R (A_1, A_2, \ldots, A_n) \]
  \end{block}
  \begin{tabular}{ c l c } \toprule
    {\bf Κωδικός} & {\bf Όνομα} & {\bf Εξάμηνο} \\ \midrule 
    504 & Βάσεις Δεδομένων & 5 \\ 
    404 & Μακροοικονομική Θεωρία ΙΙ & 4 \\  
    303 & Προγραμματισμός Υπολογιστών Ι & 3 \\   
    604 & Πληροφοριακά Συστήματα Διοίκησης & 6 \\  \bottomrule
  \end{tabular}
  \begin{exampleblock}{}%{Παράδειγμα}
    Το σύνολο $\{ \text{Κωδικός}, \text{Όνομα}, \text{Εξάμηνο} \}$
    είναι το σχήμα της σχέσης Μαθήματα.
    \vspace{-4mm}
    {\bf\color{blue} \[ \text{Μαθήματα(\underline{Κωδικός}, Όνομα, Εξάμηνο)} \] }
  \end{exampleblock}
\end{minipage}   
\end{frame}


\begin{frame}[t, fragile, shrink]
\frametitle{Στιγμιότυπο σχέσης}
  \begin{block}{Στιγμιότυπο σχέσης}
    Στιγμιότυπο σχέσης που συμβολίζεται με $t[R]$ είναι το σύνολο όλων των πλειάδων μιας σχέσης
    μια συγκεκριμένη χρονική στιγμή.
 \end{block}
  \begin{tabular}{ c l c } \toprule
    {\bf Κωδικός} & {\bf Όνομα} & {\bf Εξάμηνο} \\ \midrule 
    504 & Βάσεις Δεδομένων & 5 \\ 
    404 & Μακροοικονομική Θεωρία ΙΙ & 4 \\  
    303 & Προγραμματισμός Υπολογιστών Ι & 3 \\   
    604 & Πληροφοριακά Συστήματα Διοίκησης & 6 \\  \bottomrule
  \end{tabular} 
\end{frame}


\begin{frame}[t, fragile, shrink]
\frametitle{Γνώρισμα σχέσης}
  \begin{block}{Γνώρισμα}
    Γνώρισμα της σχέσης (πεδίο ή στήλη ενός πίνακα) είναι μια ιδιότητα της σχέσης
    και έχει ένα μοναδικό όνομα μέσα στη σχέση.
  \end{block}
\begin{minipage}{0.94\textwidth}
  \begin{tabular}{ c l c } \toprule
    {\bf Κωδικός} & {\bf Όνομα} & {\bf Εξάμηνο} \\ \midrule 
    504 & Βάσεις Δεδομένων & 5 \\ 
    404 & Μακροοικονομική Θεωρία ΙΙ & 4 \\  
    303 & Προγραμματισμός Υπολογιστών Ι & 3 \\   
    604 & Πληροφοριακά Συστήματα Διοίκησης & 6 \\  \bottomrule
  \end{tabular}
  \begin{exampleblock}{Παράδειγμα}
    Ο {\bf Κωδικός}, το {\bf Όνομα} και το {\bf Εξάμηνο} του μαθήματος
    είναι γνωρίσματα της σχέσης.
  \end{exampleblock}  
\end{minipage}  
\end{frame}


\begin{frame}[t, fragile]
\frametitle{Πεδίο ορισμού γνωρίσματος σχέσης}
\begin{minipage}{\wE}
  \begin{block}{Πεδίο ορισμού}
    Πεδίο ορισμού $dom(A_i)$ ενός γνωρίσματος $(A_i)$ είναι όλες οι επιτρεπτές τιμές
    του γνωρίσματος $A_i$.
  \end{block}
  \begin{tabular}{ c l c } \toprule
    {\bf Κωδικός} & {\bf Όνομα} & {\bf Εξάμηνο} \\ \midrule 
    504 & Βάσεις Δεδομένων & 5 \\ 
    404 & Μακροοικονομική Θεωρία ΙΙ & 4 \\  
    303 & Προγραμματισμός Υπολογιστών Ι & 3 \\   
    604 & Πληροφοριακά Συστήματα Διοίκησης & 6 \\  \bottomrule
  \end{tabular}
  \begin{exampleblock}{Παράδειγμα}
    Πεδίο ορισμού του γνωρίσματος {\bf Εξάμηνο} είναι το σύνολο των ακεραίων αριθμών
    $\{1,2,3,4,5,6,7,8\}$.
  \end{exampleblock}
\end{minipage}  
\end{frame}


\begin{frame}[t, fragile, shrink]
\frametitle{Συστοιχία ή πλειάδα}
\begin{minipage}{\wE}
  \begin{block}{Συστοιχία ή πλειάδα}
    {\cee Συστοιχία ή πλειάδα} είναι μια διατεταγμένη λίστα από τιμές  $t=\,\,\,<v_1,v_2,\ldots,v_n>$,
    που κάθε μία ανήκει στο πεδίο
    ορισμού $dom(A_i)$ του αντίστοιχου γνωρίσματος $A_i$.
  \end{block}
  \begin{tabular}{ c l c } \toprule
    {\bf Κωδικός} & {\bf Όνομα} & {\bf Εξάμηνο} \\ \midrule 
    504 & Βάσεις Δεδομένων & 5 \\   \bottomrule
  \end{tabular}
  \bigskip \par Η διατεταγμένη λίστα τιμών 
    \[ t =\,\,\, <504, {\el\text{Βάσεις Δεδομένων}}, 5> \] 
     είναι μια συστοιχία ή πλειάδα της σχέσης.
\end{minipage}
\end{frame}


\begin{frame}[t, fragile, shrink]
\frametitle{Ορισμός σχέσης}
\begin{minipage}{\wE}
  \begin{block}{Σχέση}
    Είναι ο συνδυασμός του σχήματος $R$ και του στιγμιότυπου $r$ της σχέσης. 
  \end{block}
  \pause
  \begin{exampleblock}{Σχέση}
    Γράφουμε $r(R)$ και διαβάζουμε:\\
    \begin{itemize}
      \item Μια σχέση $r$ πάνω στο σχήμα $R$.
      \item Στιγμυότυπο $r$ του (σχεσιακού) σχήματος $R$.
    \end{itemize}

  \end{exampleblock}
\end{minipage}
\end{frame}


\begin{frame}[t, fragile, shrink]
\frametitle{Βαθμός σχέσης}
\begin{minipage}{\wE}
  \begin{block}{Βαθμός σχέσης}
    Βαθμός μιας σχέσης $r(R)$ είναι το πλήθος των γνωρισμάτων της σχέσης.
  \end{block}
  \pause
  \begin{exampleblock}{Μία σχέση με βαθμό 3}
    \begin{tabular}{ c l c } \toprule
        {\bf Κωδικός} & {\bf Όνομα} & {\bf Εξάμηνο} \\ \midrule
        504 & Βάσεις Δεδομένων & 5 \\
        404 & Μακροοικονομική Θεωρία ΙΙ & 4 \\
        303 & Προγραμματισμός Υπολογιστών Ι & 3 \\
        604 & Πληροφοριακά Συστήματα Διοίκησης & 6 \\ \bottomrule
    \end{tabular}    
  \end{exampleblock}
\end{minipage}  
\end{frame}


\begin{frame}[t, fragile, shrink]
\frametitle{Πληθικότητα σχέσης}
\begin{minipage}{\wE}
  \begin{block}{Πληθικότητα σχέσης}
    Πληθικότητα μιας σχέσης $r(R)$ είναι το πλήθος των πλειάδων της σχέσης.
  \end{block}
  \pause
  \begin{exampleblock}{Μία σχέση με πληθικότητα 4}
    \begin{tabular}{ c l c } \toprule
        {\bf Κωδικός} & {\bf Όνομα} & {\bf Εξάμηνο} \\ \midrule
        504 & Βάσεις Δεδομένων & 5 \\
        404 & Μακροοικονομική Θεωρία ΙΙ & 4 \\
        303 & Προγραμματισμός Υπολογιστών Ι & 3 \\
        604 & Πληροφοριακά Συστήματα Διοίκησης & 6 \\ \bottomrule
    \end{tabular}    
  \end{exampleblock} 
\end{minipage}  
\end{frame}


\begin{frame}[t, fragile, shrink]
\frametitle{Σχήμα της βάσης} 
\begin{minipage}{\wE}
  \begin{block}{Σχήμα της βάσης δεδομένων}
    Είναι το σύνολο των σχέσεων που αποτελούν τη βάση δεδομένων.
  \end{block}
    \pause
    \begin{exampleblock}{Παράδειγμα}
      \par Μαθήματα(Κωδικός, Όνομα, Εξάμηνο) \\ 
      \par \bigskip Αίθουσες(Κωδικός, Όνομα, Χωρητικότητα) \\
      \par \bigskip Πρόγραμμα(ΚωδΜαθ, ΚωδΑιθ, Ημέρα, Ώρα) \\
  \end{exampleblock}
\end{minipage}
\end{frame}



\subsection[\textgreek{Ιδιότητες}]{\textgreek{Οι 4 βασικές ιδιότητες των σχέσεων}}


\begin{frame}[t, fragile, shrink]
\frametitle{Ιδιότητες των σχέσεων}
\begin{minipage}{\wE}
\pause
\begin{enumerate} [<+->] \itemsep 6pt
  \item {\cee Μοναδικότητα πλειάδων.} Σε μια σχέση, όλες οι πλειάδες (συστοιχίες) είναι μοναδικές.
        Δεν υπάρχουν επαναλαμβανόμενες πλειάδες.
  \item {\cee Διάταξη πλειάδων.} Δεν υπάρχει συγκεκριμένη διάταξη (ταξινόμηση) των πλειάδων σε μια σχέση.
  \item {\cee Διάταξη γνωρισμάτων.} Δεν υπάρχει επίσης, διάταξη των γνωρισμάτων μιας σχέσης.
        Τα γνωρίσματα δεν είναι διατεταγμένα πχ, από τα αριστερά προς τα δεξιά.
  \item {\cee Ατομικότητα.} Κάθε γνώρισμα έχει μια μόνο τιμή σε μια συγκεκριμένη πλειάδα.
\end{enumerate}
\end{minipage}
\end{frame}


\begin{frame}[t, fragile, shrink]
\frametitle{Μοναδικότητα}
\begin{tabular}{ c l l l } \hline 
	{\bf ΑΦΜ} & {\bf Επώνυμο} & {\bf Επάγγελμα} & {\bf Διεύθυνση}\\ \hline 
	504341 & Αρτέμης     & Μηχανικός & Δημοκρατίας 22 \\ 
	423404 & Μακροπούλου & Εκπαιδευτικός & Δημοκρατίας 22   \\ 
	348753 & Σταυρίδης   & Δημοσιογράφος & Δημοκρατίας 22 \\ 
	356712 & Παυλίδη     & Δημοσιογράφος & Δημοκρατίας 22 \\ 
	967424 & Μακροπούλου & Εκπαιδευτικός & Δημοκρατίας 22   \\  \hline 
\end{tabular}
\end{frame}


\begin{frame}[t, fragile, shrink]
\frametitle{Μοναδικότητα πλειάδων}
\begin{minipage}{\wE}
\pause
\begin{itemize} [<+->] \itemsep 4pt
  \item Με τον όρο μοναδικότητα υπονοείται πως ένα σύνολο τιμών (μια πλειάδα) 
        δεν μπορεί να επαναληφθεί μέσα σε μια σχέση.
  \item Πιθανά να επαναληφθεί ένα υποσύνολο τιμών για κάποια γνωρίσματα, όχι όμως
        το σύνολο των τιμών.
  \item Η ιδιότητα της μοναδικότητας εξασφαλίζει την ύπαρξη του {\crr πρωτεύοντος κλειδιού}.
  \item Τις περισσότερες φορές βέβαια, ένα υποσύνολο των γνωρισμάτων της σχέσης
        είναι αρκετό να ορίσει το πρωτεύον κλειδί.
  \item Τέτοιο για παράδειγμα μπορεί να είναι ο αριθμός κυκλοφορίας ενός αυτοκινήτου,
        το όνομα χρήστη μιας υπηρεσίας ηλεκτρονικού ταχυδρομείου,
        ή το ΑΦΜ ενός φορολογούμενου.
\end{itemize}
\end{minipage}
\end{frame}



\begin{frame}[t, fragile, shrink]
\frametitle{Η ταξινόμηση δεν παίζει ρόλο}
\begin{tabular}{ l r } \hline 
  {\bf Επώνυμο} & {\bf Ποσό} \\ \hline 
  Δημητριάδης & 130.50 \\ 
  Θεοδώρου    & 184.00   \\
  Λιάκος      & 390.10 \\ 
  Μαρινάκη    & 45.90 \\ 
  Τάλλος      & 129.30   \\  \hline
\end{tabular}
\hspace*{1cm}
\begin{tabular}{ l r } \hline 
  {\bf Επώνυμο} & {\bf Ποσό} \\ \hline 
  Λιάκος      & 390.10 \\ 
  Θεοδώρου    & 184.00 \\ 
  Δημητριάδης & 130.50 \\ 
  Τάλλος      & 129.30 \\ 
  Μαρινάκη    & 45.90  \\  \hline 
\end{tabular}
\end{frame}


\begin{frame}
\frametitle{Διάταξη πλειάδων}
\begin{minipage}{\wE}
  \pause
  \begin{enumerate}   \itemsep 4pt % [<+->]
    \item Δεν έχει νόημα
          να μιλάμε για την πρώτη ή την έβδομη πλειάδα μιας σχέσης.
    \item Κάθε πλειάδα μιας σχέσης μπορεί να ταυτοποιηθεί με βάση την τιμή του κλειδιού της,
          και όχι με βάση τη θέση της σε ένα σύνολο.
    \item Πχ ενδιαφέρει ο πελάτης με ΑΦΜ 004329439 και όχι ο πελάτης στην πέμπτη
          γραμμή του πίνακα πελατών.
    \item Μια πλειάδα προσδιορίζεται με βάση την τιμή κάποιου γνωρίσματος (για παράδειγμα
          την τιμή του πρωτεύοντος κλειδιού).
  \end{enumerate} 
\end{minipage}
\end{frame}




\begin{frame}
\frametitle{Διάταξη γνωρισμάτων}
Όπως και οι πλειάδες, έτσι και τα γνωρίσματα μιας σχέσης, δεν έχουν διάταξη. 
Δεν έχει σημασία πιο είναι πρώτο, δεύτερο κτλ.
  \begin{tabular}{ c c l } \toprule
    {\bf Κωδικός} & {\bf Εξάμηνο} & {\bf Όνομα} \\ \midrule 
    504 & 5 & Βάσεις Δεδομένων  \\ 
    404 & 4 & Μακροοικονομική Θεωρία ΙΙ  \\  
    303 & 3 & Προγραμματισμός Υπολογιστών Ι  \\   
    604 & 6 & Πληροφοριακά Συστήματα Διοίκησης \\  \bottomrule
  \end{tabular}
  \par
  \begin{minipage}{\wE}
    \begin{exampleblock}{Παράδειγμα}
      Το παραπάνω σχήμα απεικονίζει τη σχέση {\ra\el μάθημα}
      με διαφορετική σειρά εμφάνισης των γνωρισμάτων της. Αν τα δύο σχήματα ειδωθούν ως σχέσεις,
      τότε απεικονίζουν δύο πανομοιότυπες σχέσεις, δεν υπάρχει καμία διαφορά!
    \end{exampleblock}
  \end{minipage}
\end{frame}


\begin{frame}[t, fragile, shrink]
\frametitle{Ατομικότητα και 1\textsuperscript{η} κανονική μορφή}
\begin{minipage}{\wE}
  \begin{block}{Ατομικότητα}
    Ο όρος ατομικότητα των τιμών αναφέρεται στη μη διάσπασή τους σε απλούστερες τιμές. 
    Αναφέρεται επίσης
    στο γεγονός πως κάθε πλειάδα μιας σχέσης έχει μόνο μία τιμή σε κάθε γνώρισμα.
\end{block}
\color{red}
		\begin{tabular}{ c c } \toprule 
		{\bf Πελάτης} & {\bf Παραγγελία} \\ \midrule 
		109 & 5018 \\
		163 & 4012, 5901 \\
		180 & 4291, 3103 \\ \bottomrule
                    & \\
                    & \\
                    \multicolumn{2}{l} {1\textsuperscript{η} ΚΜ? Όχι} \\
		\end{tabular}
\color{blue}		
\pause \hspace*{1cm}
		\begin{tabular}{ c c }  \toprule 
		{\bf Πελάτης} & {\bf Παραγγελία} \\ \midrule  
		109 & 5018 \\
		163 & 4012 \\
		163 & 5901 \\
		180 & 4291 \\
		180 & 3103 \\ \bottomrule 
                    \multicolumn{2}{l} {1\textsuperscript{η} ΚΜ? Ναι} \\
		\end{tabular}

\end{minipage}
\end{frame}


\begin{frame}[t, fragile, shrink]
\frametitle{Ατομικότητα -- συνέχεια}
\begin{minipage}{\wE}
  \begin{tabular}{ c l l} \hline
   {\bf Φανέλα} & {\bf Όνομα} & {\bf Επώνυμο}	\\ \hline
       3 & Μάριος & Αλεξίου \\ 
       4 & Δημήτρης-Άγγελος & Σταθόπουλος \\ 
      11 & Βασίλης & Μαργαρίτης \\ 
       7 & Αλέξανδρος & Παπαβασιλείου \\ 
      19 & Βασίλης & Βλάχος \\ 	\hline					
  \end{tabular}
\\
\bigskip
\par
Είναι η σχέση σε πρώτη κανονική μορφή? 
Είναι δηλαδή όλες οι τιμές όλων των γνωρισμάτων ατομικές?
\end{minipage}
\end{frame}


\subsection[\textgreek{Είδη}]{\textgreek{Τα είδη των σχέσεων}}

\begin{frame}[t, fragile, shrink]
\frametitle{Είδη σχέσεων: Επώνυμες σχέσεις}
\begin{minipage}{\wE}
\begin{block}{Επώνυμες σχέσεις}
{\crr Επώνυμες} σχέσεις είναι αυτές που έχουν οριστεί από το 
{\cbb Σχεσιακό Σύστημα Διαχείρισης Βάσεων Δεδομένων} και έχουν κάποιο
μοναδικό όνομα στη βάση δεδομένων. Για παράδειγμα, οι πίνακες και όψεις που ορίζονται με τις
εντολές της {\sq SQL}: {\sq CREATE TABLE} και {\sq CREATE VIEW} είναι επώνυμες σχέσεις.
Είναι δουλειά του {\bb Σχεσιακού Συστήματος Διαχείρισης Βάσεων Δεδομένων} να 
ελέγχει την εγκυρότητα του ορισμού και τη μοναδικότητα
του ονόματος. Μια επώνυμη σχέση, μπορεί στη συνέχεια να κληθεί με το όνομά της 
\end{block}
\end{minipage}
\end{frame}


\begin{frame}[t, fragile, shrink]
\frametitle{Είδη σχέσεων: Παραστάσιμες σχέσεις}
\begin{minipage}{\wE}
  \begin{block}{Παραστάσιμες σχέσεις}
    {\crr Παραστάσιμες} είναι οι σχέσεις που προκύπτουν από σχεσιακές παραστάσεις επώνυμων
    σχέσεων. Κάθε επώνυμη σχέση είναι παραστάσιμη, μια παραστάσιμη σχέση ωστόσο δεν είναι
    υποχρεωτικά επώνυμη.
  \end{block}
\end{minipage}
\end{frame}


\begin{frame}[t, fragile, shrink]
\frametitle{Είδη σχέσεων: Παράγωγες σχέσεις}
\begin{minipage}{\wE}
  \begin{block}{Παράγωγες σχέσεις}
    {\crr Παράγωγες} είναι οι επώνυμες σχέσεις που ορίζονται με τη βοήθεια άλλων επώνυμων σχέσεων.
    Οι παράγωγες σχέσεις είναι παραστάσιμες, χωρίς να ισχύει υποχρεωτικά το αντίθετο.
  \end{block}
\end{minipage}
\end{frame}


\begin{frame}[t, fragile, shrink]
\frametitle{Είδη σχέσεων: Βασικές σχέσεις}
\begin{minipage}{\wE}
  \begin{block}{Βασικές σχέσεις}
    {\crr Βασικές} είναι οι επώνυμες σχέσεις που δεν είναι παράγωγες, δηλαδή ορίζονται αυτόνομα
    από άλλες σχέσεις. Κάθε βάση δεδομένων έχει τουλάχιστον μία βασική
    σχέση. Στην πράξη, οι βασικές σχέσεις είναι οι μόνες που αποθηκεύουν δεδομένα. Επομένως
    είναι και οι πιο βασικές!
  \end{block}
  \begin{exampleblock}{Παράδειγμα}
    Οι πίνακες που ορίζονται με την εντολή {\sq CREATE TABLE}
    είναι βασικοί πίνακες (βασικές σχέσεις). 
  \end{exampleblock}
\end{minipage}
\end{frame}


\begin{frame}[t, fragile, shrink]
\frametitle{Είδη σχέσεων: Όψεις}
\begin{minipage}{\wE}
  \begin{block}{Όψεις}
    {\crr Όψεις} (αλλιώς και απόψεις) είναι οι επώνυμες παράγωγες σχέσεις. Ο ορισμός τους 
    στηρίζεται στην ύπαρξη μιας τουλάχιστον βασικής σχέσης. Οι όψεις είναι επώνυμες σχέσεις, 
    με την {\sq SQL} δημιουργούνται με την εντολή {\sq CREATE VIEW}. Οι όψεις δεν αποθηκεύουν
    δεδομένα, γι' αυτό λέγεται και ιδεατοί πίνακες. Μια όψη μπορεί να οριστεί με βάση κάποια άλλη
    όψη, ωστόσο, κάπου στην άκρη του νήματος, πρέπει να υπάρχει μια βασική σχέση.
  \end{block}
\end{minipage}
\end{frame}


\begin{frame}[t, fragile, shrink]
\frametitle{Είδη σχέσεων: Ενδιάμεσα αποτελέσματα}
\begin{minipage}{\wE}
  \begin{block}{Ενδιάμεσα αποτελέσματα}
    {\crr Ενδιάμεσα αποτελέσματα} είναι οι σχέσεις που παράγονται σε ενδιάμεσα στάδια πολύπλοκων
    ερωτημάτων. Τα ενδιάμεσα αποτελέσματα έχουν πρόσκαιρη μόνο ύπαρξη στη βάση δεδομένων.
  \end{block}
\end{minipage}
\end{frame}


\begin{frame}[t, fragile, shrink]
\frametitle{Είδη σχέσεων: Αποτελέσματα ερωτημάτων}
\begin{minipage}{\wE}
  \begin{block}{Αποτελέσματα ερωτημάτων}
    {\crr Αποτελέσματα ερωτημάτων} είναι οι ανώνυμες παράγωγες σχέσεις που δημιουργούνται κατά
    την εκτέλεση ερωτημάτων και την προβολή των αποτελεσμάτων. 
    Τα αποτελέσματα ερωτημάτων έχουν παροδική ύπαρξη στις βάσεις δεδομένων. 
    Για να κρατηθούν τα αποτελέσματα στη βάση
    πρέπει το ερώτημα να γίνει επώνυμη σχέση, δηλαδή όψη.
  \end{block}
\end{minipage}
\end{frame}


\begin{frame}[t,fragile]
\frametitle{Η ερμηνεία και το κατηγόρημα μιας σχέσης}
\begin{minipage}{\wE}
\pause
\begin{itemize} [<+->] \itemsep 6pt
 \item  Το σχήμα μιας σχέσης έχει ένα νόημα, ή αλλιώς μια ερμηνεία, 
       που μπορεί να εκληφθεί ως παράσταση αληθείας
 \item  Το νόημα κάθε σχέσης μιας βάσης δεδομένων πρέπει να είναι γνωστό στους χρήστες
 \item Το κατηγόρημα μπορεί να εκτιμηθεί ως {\sq TRUE} ή {\sq FALSE}, 
       ανάλογα με το στιγμιότυπο της σχέσης
 \item  Για παράδειγμα, για τη σχέση 
\emph{Υπάλληλος (Κωδικός, Όνομα, Επώνυμο, Τμήμα, Μισθός, Ημερ.Πρόσληψης)}
κατηγόρημα είναι μια πρόταση, όπως:
\textit{Ο υπάλληλος με κωδικό 243, έχει Όνομα Δέσποινα, και Επώνυμο Παπαδοπούλου, 
και εργάζεται στο Τμήμα με κωδικό 2,
και αμείβεται με Μισθό 1609.52 \euro, και προσλήφθηκε στις 5/3/1999 και δεν υπάρχει άλλος υπάλληλος με ακριβώς
τον ίδιο κωδικό.}
\end{itemize}
\end{minipage}
\end{frame}
