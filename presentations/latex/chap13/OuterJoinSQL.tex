\section[]{\textgreek{Εξωτερική σύζευξη πινάκων στην \textlatin{SQL}}}


\begin{frame}[t, fragile, shrink]
\frametitle{Παράδειγμα εξωτερικής σύζευξης -- το πρόβλημα}
\begin{minipage}{\wE}
\vspace*{-0.5cm}
\begin{exampleblock}{\small Να βρεθούν τα ονόματα των υπαλλήλων του τμήματος 4
              και οι κωδικοί των έργων που συμμετέχουν}
\en
\[
\begin{array}{l}
  \Pi_{e.firstname, e.lastname, w.proid}
  (\sigma_{e.depid=4}                    \\
    ( \varrho_{e} (employees) \bowtie_{d.depid=e.depid} \varrho_{w} (workson) )
  )
\end{array}
\]
\pause
\vspace*{-0.5cm}
\begin{SQL}
  SELECT e.firstname, e.lastname, w.proid
    FROM employees e INNER JOIN workson w
         ON e.empid = w.empid
   WHERE e.depid = 4;
\end{SQL}
\el
\end{exampleblock}
\small
\begin{columns}
\begin{column}{0.55\linewidth}
\scalebox{0.9}{
\begin{tabular}{l l r} \toprule
{\bb\en firstname} & {\bb\en lastname} & {\bb\en proid} \\  \midrule
Νίκος & Βλάχος & 12 \\
Βαγγέλης & Χριστόπουλος & 12 \\
Βαγγέλης & Χριστόπουλος & 14 \\
Βαγγέλης & Χριστόπουλος & 38 \\
Παύλος & Περίδης & 43 \\ \bottomrule
\end{tabular}
}
\end{column}
\begin{column}{0.5\linewidth}
  \par \color{red} Τι γίνεται με τους υπαλλήλους του τμήματος 4 που δεν απασχολούνται σε κανένα έργο?
\end{column}
\end{columns}
\end{minipage}
\end{frame}


\begin{frame}[t, fragile, shrink]
\frametitle{Παράδειγμα εξωτερικής σύζευξης -- Η λύση}
\begin{minipage}{\wE}
\small
\vspace*{-0.5cm}
\begin{exampleblock}{\small Υπάλληλοι του τμήματος 4
              και οι κωδικοί των έργων}
\[
\begin{array}{l}
  \Pi_{e.firstname, e.lastname, w.proid}
  (\sigma_{e.depid=4}                    \\
    ( \varrho_{e}(employees) \lojoin_{e.empid = w.empid} \varrho_{w}(workson) )
  )
\end{array}
\]
\pause
\vspace*{-0.5cm}
\en
\begin{SQL}
  SELECT e.firstname, e.lastname, w.proid
    FROM employees e LEFT JOIN workson w
         ON e.empid = w.empid
   WHERE e.depid = 4;
\end{SQL}
\end{exampleblock}
\el
\small
\pause
\begin{columns}[T]
\begin{column}{0.55\linewidth}
\vspace{-0.4cm}
\scalebox{0.9}{
\begin{tabular}{l l r} \toprule
{\bb\en firstname} & {\bb\en lastname} & {\bb\en proid} \\  \midrule
Νίκος & Βλάχος & 12 \\
Βαγγέλης & Χριστόπουλος & 12 \\
Βαγγέλης & Χριστόπουλος & 14 \\
Βαγγέλης & Χριστόπουλος & 38 \\
\rowcolor{yellow!50!brown}
Νίκος & Στεργιόπουλος & {\sq NULL} \\
Παύλος & Περίδης & 43 \\
\rowcolor{yellow!50!brown}
Ευθαλεία & Μικράκη & {\sq NULL} \\ \bottomrule
\end{tabular}
}
\end{column}
\begin{column}{0.4\linewidth}
\small
  \par Η στήλη {\ra proid} συμπληρώνεται με {\sq NULL}
       για τους υπαλλήλους που δεν απασχολούνται σε κανένα έργο.
\end{column}
\end{columns}
\end{minipage}
\end{frame}


\begin{frame}[t, fragile, shrink]
\frametitle{Αριστερή και δεξιά σύζευξη -- Ισοδυναμία}
\begin{exampleblock}{Αριστερή σύζευξη}
\en
\[
\begin{array}{l}
  \Pi_{e.firstname, e.lastname, w.proid}
  (\sigma_{e.depid=4}                    \\
    ( \varrho_{e}(employees) \lojoin_{e.empid = w.empid} \varrho_{w}(workson) )
  )
\end{array}
\]
\begin{SQL}
  SELECT e.firstname, e.lastname, w.proid
    FROM employees e {&color{red}&bf LEFT JOIN} workson w
         ON e.empid = w.empid
   WHERE e.depid = 4;
\end{SQL}
\end{exampleblock}
\el
\begin{exampleblock}{Δεξιά σύζευξη}
\en
\[
\begin{array}{l}
  \Pi_{e.firstname, e.lastname, w.proid}
  (\sigma_{e.depid=4}                    \\
    ( \varrho_{w}(workson) \rojoin_{w.empid = e.empid}   \varrho_{e}(employees) )
  )
\end{array}
\]
\begin{SQL}
  SELECT e.firstname, e.lastname, w.proid
    FROM workson w {&color{red}&bf RIGHT JOIN} employees e
         ON w.empid = e.empid
   WHERE e.depid = 4;
\end{SQL}
\end{exampleblock}
\end{frame}


\begin{frame}[t, fragile, shrink]
\el
\frametitle{Αριστερή και δεξιά σύζευξη}
\large
\begin{enumerate} \itemsep 6pt % [<+->] \pause 
\item Οι δύο προτάσεις {\sq SQL} είναι απολύτως {\bb ισοδύναμες}
και δίνουν το ίδιο αποτέλεσμα.

\item Κατά παράδοση, προτιμάται η {\bb αριστερή} σύζευξη.

\item
Οι εξωτερικές συζεύξεις χρησιμοποιούνται όταν  θέλουμε στο
αποτέλεσμα όλες τις εγγραφές
ενός πίνακα, ανεξάρτητα αν αυτές έχουν σύνδεση με τον άλλο πίνακα
που υπάρχει στη σύζευξη.

\item
Κατά την εξωτερική σύζευξη, αν υπάρχουν  μη συνδεδεμένες εγγραφές,
δημιουργούνται τιμές {\sq NULL}.

\item
Ο έλεγχος ({\sq WHERE}) για τιμές {\sq NULL}
είναι πολύ συχνός στις εξωτερικές συνδέσεις.

\end{enumerate}
\end{frame}


\begin{frame}[t, fragile, shrink]
\frametitle{Εξωτερική σύζευξη και έλεγχος {\en NULL}}
\begin{minipage}{\wE}
\vspace*{-0.5cm}
\begin{exampleblock}{\small Ποιοι υπάλληλοι δεν απασχολούνται σε κανένα έργο?}
\en
\[
\begin{array}{l}
  \Pi_{e.*}
  (\sigma_{w.empid \text{ IS NULL}}                    \\
    ( \varrho_{e}(employees) \lojoin_{e.empid = w.empid} \varrho_{w}(workson) )
  )
\end{array}
\]
\pause
\en
\begin{SQL}
  SELECT e.*
    FROM employees e LEFT JOIN workson w
                       ON e.empid = w.empid
   WHERE w.empid IS NULL;
\end{SQL}
\el
\end{exampleblock}
\small
\begin{enumerate}
  \item Το ερώτημα δε μπορεί να απαντηθεί με φυσική ή εσωτερική σύζευξη.
  \item Οι τιμές των πεδίων {\ra e.empid} και {\ra w.empid}   είτε ταυτίζονται,
        είτε κάποια τιμή   του πεδίου {\ra e.empid}  δεν έχει    αντίστοιχη
        τιμή στον πίνακα {\ra workson}.
\end{enumerate}
\end{minipage}
\end{frame}


\begin{frame}[t, fragile, shrink]
\frametitle{Εξωτερική σύζευξη -- επιπλέον ανάλυση}
\begin{minipage}{\wE}
\vspace{-0.5cm}
\begin{exampleblock}{\small Ποιοι υπάλληλοι δεν απασχολούνται σε κανένα έργο?}
\en
\begin{SQL}
  SELECT e.empid, w.empid
    FROM employees e LEFT JOIN workson w
         ON e.empid = w.empid
   WHERE w.empid IS NULL;
\end{SQL}
\end{exampleblock}
\el
\vspace{-0.2cm}
\pause
\begin{columns}
\begin{column}{0.4\linewidth}
\begin{tabular}{r r}  \toprule
{\ra e.empid} & {\ra w.empid} \\ \midrule
205 & {\en NULL} \\
311 & {\en NULL} \\
342 & {\en NULL} \\
... & ... \\ \bottomrule
\end{tabular}
\end{column}
\begin{column}{0.55\linewidth}
\begin{itemize} \pause
  \item 205: υπάρχει στον πίνακα {\ra employees}
        αλλά δεν υπάρχει στον πίνακα {\ra workson}.
  \item Η αριστερή σύζευξη επιτρέπει  την εμφάνιση της τιμής  205
        στο  πεδίο {\ra e.empid}.
  \item Το πεδίο {\ra w.empid} θα συμπληρωθεί  με {\sq NULL}.
\end{itemize}
\end{column}
\end{columns}
\end{minipage}
\end{frame}


\begin{frame}[t, fragile, shrink]
\frametitle{Εξωτερική σύζευξη -- σύνοψη}
\begin{minipage}{\wE}
\pause
\begin{enumerate} \itemsep 4pt % [<+->]
  \item Κάποιες εγγραφές του πίνακα {\ra employees} δεν έχουν ταιριαστές
        εγγραφές στον πίνακα {\ra workson}.
  \item Ο πίνακας {\ra workson} δεν περιέχει καμία εγγραφή με {\sq NULL} τιμές
        Ο τρόπος με τον οποίο έγινε η σύζευξη 
        των πινάκων παρήγαγε τις τιμές {\sq NULL}.
  \item Η αριστερή εξωτερική σύζευξη ορίζει πως
        στο αποτέλεσμα θα υπάρχουν όλες οι εγγραφές του αριστερού πίνακα και
        στα πεδία του δεξιού πίνακα θα τοποθετηθούν είτε τιμές που αντιστοιχούν
        στον κανόνα της σύζευξης είτε τιμές {\sq NULL},
        εκεί όπου δεν βρέθηκαν ταιριαστές εγγραφές.
  \item Επομένως οι  εγγραφές με {\sq NULL} τιμές στα πεδία του   πίνακα {\ra workson},
        δεν οφείλονται σε αποθηκευμένες τιμές   αλλά σε παραγόμενες μετά από
        εξωτερική σύζευξη.
\end{enumerate}
\end{minipage}
\end{frame}


\begin{frame}[t, fragile, shrink]
\frametitle{Εξωτερική σύζευξη 1:1}
\begin{minipage}{\wE}
\vspace{-0.5cm}
\begin{exampleblock}{\small Να βρεθεί ο κωδικός και το όνομα των υπαλλήλων που δεν είναι διευθυντές}
\en
\begin{SQL}
  SELECT e.empid, e.firstname, e.lastname
    FROM employees e LEFT JOIN departments d
         ON e.empid = d.manager   
   WHERE d.manager IS NULL;
\end{SQL}
\end{exampleblock}
\el
\vspace{-0.2cm}
\pause
\begin{alertblock}{\small Να βρεθεί ο κωδικός και το όνομα των υπαλλήλων που δεν είναι διευθυντές}
\en
\begin{SQL}
  SELECT e.empid, e.firstname, e.lastname
    FROM employees e INNER JOIN departments d
         ON e.empid = d.manager
   WHERE d.manager IS NULL;
\end{SQL}
\end{alertblock}
\end{minipage}
\end{frame}


\begin{frame}[t, fragile, shrink]
\frametitle{Εξωτερική ή εσωτερική σύζευξη?}
\begin{minipage}{\wE}
\vspace{-0.5cm}
\begin{block}{\footnotesize Να βρεθεί ο κωδικός και το όνομα των υπαλλήλων που {\bf είναι} διευθυντές}
\en
\begin{SQL}
  SELECT e.empid, e.firstname, e.lastname
    FROM employees e LEFT JOIN departments d
         ON e.empid = d.manager
   WHERE d.manager IS NOT NULL;
\end{SQL}
\end{block}
\el
\vspace{-0.2cm}
\pause
\begin{block}{\footnotesize  Να βρεθεί ο κωδικός και το όνομα των υπαλλήλων που {\bf είναι}  διευθυντές}
\en
\begin{SQL}
  SELECT e.empid, e.firstname, e.lastname
    FROM employees e INNER JOIN departments d
         ON e.empid = d.manager;
\end{SQL}
\end{block}
\par {\crr Σχολιάστε διαφορές και ομοιότητες στο αποτέλεσμα.}
\end{minipage}
\end{frame}
