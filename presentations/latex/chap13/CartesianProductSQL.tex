\section[]{\textgreek{Καρτεσιανό γινόμενο και θ σύζευξη}}



\begin{frame}[t, fragile, shrink]
\frametitle{Καρτεσιανό γινόμενο}
\begin{minipage}{\wE}
\begin{block}{Καρτεσιανό γινόμενο:}
Ο συνδυασμός των πινάκων {\ra departments} και {\ra employees}
\[
departments \times employees
\]
λέγεται καρτεσιανό γινόμενο, και περιλαμβάνει όλους τους συνδυασμούς των
εγγραφών των πινάκων {\ra departments} και {\ra employees}.
\pause
\begin{itemize}
  \item Στο αποτέλεσμα περιλαμβάνονται όλα τα πεδία των πινάκων
        {\ra departments} και {\ra employees}.
  \item Οι εγγραφές του αποτελέσματος περιλαμβάνει όλους τους συνδυασμούς των εγγραφών
        των πινάκων {\ra departments} και {\ra employees}.
\end{itemize}  
\end{block}
\end{minipage}
\end{frame}


\begin{frame}[t, fragile, shrink]
\frametitle{Καρτεσιανό γινόμενο $departments \times employees$}
%\begin{minipage}{\wE}
\begin{itemize}
  \item Όλες οι δυνατές συνδέσεις ανάμεσα στα τμήματα και σε δείγμα 3 υπαλλήλων.
  \item Μόνο οι {\crr κόκκινες γραμμές} αντιστοιχούν στην πραγματικότητα: \\ (1, 109), (6, 102), (2,153).
  \item Τα καρτεσιανό γινόμενο απεικονίζει όλα τα {\bbl πιθανά ενδεχόμενα},
        όχι απαραίτητα τα {\bbl πραγματικά γεγονότα}.
\end{itemize}
%\end{minipage}
\en
\begin{tikzpicture}
  \tikzstyle{every state} = [fill, draw=black, blue!50, text=black]
  \tikzstyle{every state} = [inner sep=0.5pt, minimum size=0pt, scale=0.9]
   
  \def\A{0.0};
  \def\B{4.0};

  \draw (\A, 0) ellipse (22pt and 60pt); % A
  \draw (\B, 0) ellipse (22pt and 60pt); % B

  % A nodes
  \node[state] (a1) at (\A,  50pt) {1};
  \node[state] (a2) at (\A,  30pt) {2};
  \node[state] (a3) at (\A,  10pt) {3};
  \node[state] (a4) at (\A, -10pt) {4};
  \node[state] (a5) at (\A, -30pt) {5};
  \node[state] (a6) at (\A, -50pt) {6};
  
  % B nodes
  \node[state] (b1) at (\B,  30pt) {102};
  \node[state] (b2) at (\B,   0pt) {109};
  \node[state] (b3) at (\B, -30pt) {153};
  
  % connection lines
  \path[dashed] (a1) edge  (b1);
  \path[dashed] (a2) edge  (b1);
  \path[dashed] (a3) edge  (b1);
  \path[dashed] (a4) edge  (b1);
  \path[dashed] (a5) edge  (b1);
  \path[red, thick] (a6) edge  (b1);
  \path[red, thick] (a1) edge  (b2);
  \path[dashed] (a2) edge  (b2);
  \path[dashed] (a3) edge  (b2);
  \path[dashed] (a4) edge  (b2);
  \path[dashed] (a5) edge  (b2);
  \path[dashed] (a6) edge  (b2);

  \path[dashed] (a1) edge  (b3);
  \path[red, thick] (a2) edge  (b3);
  \path[dashed] (a3) edge  (b3);
  \path[dashed] (a4) edge  (b3);
  \path[dashed] (a5) edge  (b3);
  \path[dashed] (a6) edge  (b3);

  % annotate names
  \draw (\A, 65pt) node {\large departments};
  \draw (\B, 65pt) node {\large employees};

\end{tikzpicture}
\el
\end{frame}




\begin{frame}[t, fragile, shrink]
\frametitle{Καρτεσιανό γινόμενο στην {\en SQL}}
\begin{minipage}{\wE}
\begin{exampleblock}{Καρτεσιανό γινόμενο τμημάτων και υπαλλήλων}
\[ departments \times employees \]
\en
\begin{SQL}
  SELECT *
    FROM departments, employees;
\end{SQL}
\el
\end{exampleblock}
\begin{itemize}
  \item Η σύνταξη στην {\sq SQL} είναι απλή: γράφουμε τους πίνακες μετά τον όρο {\sq FROM}
        και τους χωρίζουμε με κόμμα.
  \item Μπορούμε να γράψουμε περισσότερο από δύο πίνακες.
  \item Προσοχή! το αποτέλεσμα μπορεί να περιέχει μεγάλο όγκο εγγραφών,
        πχ δύο πίνακες με 6 και 30 εγγραφές αντίστοιχα δίνουν στο αποτέλεσμα
        $6\times30=180$ εγγραφές.
\end{itemize}
\end{minipage}
\end{frame}



\begin{frame}[t, fragile, shrink]
\frametitle{Καρτεσιανό γινόμενο στην {\en SQL92, CROSS JOIN}}
\begin{minipage}{\wE}
\pause
\begin{block}{Με χρήση του όρου {\en CROSS JOIN}}
\en
\begin{SQL}
  SELECT *
    FROM departments CROSS JOIN employees;
\end{SQL}
\el
\end{block}
\pause
\begin{block}{Η απλά {\en JOIN}}
\en
\begin{SQL}
  SELECT *
    FROM departments JOIN employees;
\end{SQL}
\el
\end{block}
\pause
\begin{itemize}
  \item Οι δύο εκφράσεις είναι ισοδύναμες, θα επιστρέψουν το ίδιο αποτέλεσμα.
  \item Προτιμούμε τον πρώτο τρόπο {\sq CROSS JOIN}, δηλώνει με πιο καθαρό τρόπο τη σύζευξη
        με βάση το καρτεσιανό γινόμενο.
\end{itemize}
\end{minipage}
\end{frame}



\begin{frame}[t, fragile, shrink]
\frametitle{Αποτέλεσμα καρτεσιανού γινομένου}
\en
\begin{SQL}
  SELECT * 
    FROM departments, employees;

&scalebox{0.8}{depid  depname     manager  empid  firstname  lastname  depid  salary  hiredate} 
&scalebox{0.8}{-------------------------------------------------------------------------------}
&scalebox{0.8}{&mgr{    1  Διοίκ./Επιβ.    109     102  Νικηφόρος  Διαμαντίδης   6 1212.50 2003-06-02}}
&scalebox{0.8}{&mgr{    2  Οικον./Λογ.     153     102  Νικηφόρος  Διαμαντίδης   6 1212.50 2003-06-02}}
&scalebox{0.8}{&mgr{    3  Επιστημ./Μηχ.   431     102  Νικηφόρος  Διαμαντίδης   6 1212.50 2003-06-02}}
&scalebox{0.8}{&mgr{    4  Εξωτ. συνερ.    230     102  Νικηφόρος  Διαμαντίδης   6 1212.50 2003-06-02}}
&scalebox{0.8}{&mgr{    5  Γραμματείας     234     102  Νικηφόρος  Διαμαντίδης   6 1212.50 2003-06-02}}
&scalebox{0.8}{&mgr{    6  Μάνατζ./Πωλ.    189     102  Νικηφόρος  Διαμαντίδης   6 1212.50 2003-06-02}}
&scalebox{0.8}{&mgr{    1  Διοίκ./Επιβ.    109     109  Μαρία      Αθανασίου     1 2787.69 2000-01-26}}
&scalebox{0.8}{&mgr{    2  Οικον./Λογ.     153     109  Μαρία      Αθανασίου     1 2787.69 2000-01-26}}
&scalebox{0.8}{&mgr{    3  Επιστημ./Μηχ.   431     109  Μαρία      Αθανασίου     1 2787.69 2000-01-26}}
&scalebox{0.8}{.............................................................................}
180 rows in set (0.00 sec)
\end{SQL}
\el
\end{frame}


\begin{frame}[t, fragile, shrink]
\frametitle{Επιλογή πεδίων από πίνακες}
\begin{minipage}{\wE}
\en
\begin{SQL}
  SELECT departments.depid, depname, empid, lastname 
    FROM departments, employees;

depid  depname       empid  lastname 
---------------------------------------
&mgr{    1  Διοίκ./Επιβ.    102   Διαμαντίδης}
&mgr{    2  Οικον./Λογ.     153   Διαμαντίδης   }
&mgr{    3  Επιστημ./Μηχ.   431   Διαμαντίδης   }
&mgr{    4  Εξωτ. συνερ.    230   Διαμαντίδης   }
&mgr{    5  Γραμματείας     234   Διαμαντίδης   }
&mgr{    6  Μάνατζ./Πωλ.    189   Διαμαντίδης   }
&mgr{    1  Διοίκ./Επιβ.    109   Αθανασίου     }
&mgr{    2  Οικον./Λογ.     153   Αθανασίου     }
&mgr{    3  Επιστημ./Μηχ.   431   Αθανασίου     }
.........................................
180 rows in set (0.00 sec)
\end{SQL}
\el
\end{minipage}
\end{frame}



\begin{frame}[t, fragile, shrink]
\frametitle{Επιλογή πεδίων από πίνακες}
\begin{minipage}{\wE}
\en
\begin{SQL}
  SELECT departments.depid, depname, empid, lastname 
    FROM departments, employees;
\end{SQL}
\el
\begin{block}{Τι προσέχουμε:}
\begin{enumerate}
  \item Οποιοδήποτε όνομα πεδίου/στήλης υπάρχει στους πίνακες που ακολουθούν τον όρο {\sq FROM}
        μπορούν να τοποθετηθούν μετά τον όρο {\sq SELECT}.
  \item Σε περίπτωση που το όνομα πεδίου είναι μοναδικό σε όλους τους πίνακες 
        μπορούμε να το γράψουμε ως έχει, πχ {\ra depname} ή {\ra lastname}.
  \item Αν το ίδιο όνομα πεδίου υπάρχει σε δύο διαφορετικούς πίνακες,
        τότε πρέπει να γραφεί με τη μορφή {\sq\el πίνακας.πεδίο}.
\end{enumerate} 
\end{block} 
\end{minipage}
\end{frame}


\begin{frame}[t, fragile, shrink]
\frametitle{Μετονομασία πινάκων}
\begin{minipage}{\wE}
\begin{block}{Τι προσέχουμε:}
\en
\begin{SQL}
  SELECT d.depid, d.depname, e.empid, e.lastname 
    FROM departments d, employees e;
\end{SQL}
\el
\begin{enumerate}
  \item Η χρήση ψευδωνύμων είναι προαιρετική.
  \item Βολεύει όταν το όνομα του πίνακα γράφεται πολλές φορές.
\end{enumerate} 
\end{block}
\pause
\begin{block}{Ίδιοι κανόνες στην {\en SQL92}:}
\en
\begin{SQL}
  SELECT d.depid, d.depname, e.empid, e.lastname 
    FROM departments d CROSS JOIN employees e;
\end{SQL}
\el
\end{block}
\end{minipage}
\end{frame}




\begin{frame}[t, fragile, shrink]
\frametitle{Καρτεσιανό γινόμενο υπαλλήλων και έργων}
\begin{minipage}{\wE}
\en
\begin{SQL}
  SELECT e.empid, p.proid
    FROM employees e, projects p;

| empid | proid |
-----------------
|   102 |     5 |
|   102 |    12 |
|   102 |    14 |
|   102 |    21 |
|   102 |    38 |
|   102 |    43 |
|   109 |     5 |
|   109 |    12 |
.................
180 rows in set (0.00 sec)
\end{SQL}
\el
\end{minipage}
\end{frame}


\begin{frame}[t, fragile, shrink]
\frametitle{Καρτεσιανού γινόμενο υπαλλήλων και έργων}
\begin{minipage}{\wE}
%\begin{block}{\[ \Pi_{empid, proid} \left(\varrho_{e}(employees) \times \varrho_{p} (projects) \right) \]}
%\vspace*{-0.4cm}
\en
\begin{SQL}
  SELECT e.empid, p.proid
    FROM employees e, projects p;
\end{SQL}
\el
%\end{block}
%\footnotesize 
\pause
\begin{enumerate} %[<+->]
  \item Το αποτέλεσμα συνδέει {\crr όλους} τους υπαλλήλους με {\crr όλα} τα έργα.
  \item Είναι {\crr πιθανό} ένας υπάλληλος να απασχολείται σε όλα τα έργα,
        αλλά αυτό δεν συμβαίνει {\crr υποχρεωτικά} για όλους τους υπαλλήλους.
  \item Είναι {\crr πιθανό} ένα έργο να απασχολεί όλους τους υπαλλήλους,
        αλλά αυτό δεν συμβαίνει {\crr υποχρεωτικά} για όλα τα έργα.
  \item Το καρτεσιανό γινόμενο μας δίνει όλα τα {\crr πιθανά ενδεχόμενα}, δεν μας
        αποκαλύπτει το «τι συμβαίνει».
\end{enumerate}
\end{minipage}
\end{frame}


\begin{frame}[t, fragile, shrink]
\frametitle{Σύζευξη θήτα στην {\en SQL}}
\begin{minipage}{\wE}
\vspace*{-0.5cm}
\begin{exampleblock}{Σύζευξη τμημάτων και υπαλλήλων}
\[ departments \bowtie_{departments.depid = employees.depid} employees \]
\vspace*{-0.5cm}
\en
\begin{SQL}
  SELECT *
    FROM departments, employees
   WHERE departments.depid = employees.depid;
\end{SQL}
\el
\end{exampleblock}
\pause
\begin{enumerate} % [<+->] \pause
  \item Προσέξτε τη γραφή {\bb πίνακας.πεδίο}.
  \item Στη φράση {\sq WHERE} μπορούν να προστεθούν   επιπλέον
        περιορισμοί με λογική σύζευξη ({\sq AND}).
  \item Η θήτα σύζευξη είναι δυνατό να πραγματοποιηθεί  και με πεδία
        που δεν έχουν το ίδιο όνομα.
  \item Στο αποτέλεσμα εμφανίζονται μόνο οι εγγραφές για τις οποίες η συνθήκη είναι {\sq TRUE}.      
\end{enumerate}
\end{minipage}
\end{frame}



\begin{frame}[t, fragile, shrink]
\frametitle{θ σύζευξη, επιπλέον παράδειγμα}
\begin{minipage}{\wE}
\begin{exampleblock}{Τμήματα με υπαλλήλους με μισθό πάνω από 1500 \euro}
\en
\begin{SQL}
  SELECT DISTINCT d.depname
    FROM departments d, employees e
   WHERE d.depid = e.depid
     AND e.salary > 1500;
\end{SQL}
\el
\end{exampleblock}
\pause
\begin{enumerate} % [<+->] \pause
  \item Περιορισμός εγγραφών της θ σύζευξης με βάση μια παράσταση σύγκρισης.
  \item Στο αποτέλεσμα εμφανίζονται μόνο οι εγγραφές για τις οποίες όλες οι είναι {\sq TRUE}.
  \item Η θ σύζευξη δεν έχει νόημα να ακολουθείται από λογική διάζευξη {\sq OR}.  
\end{enumerate}
\end{minipage}
\end{frame}


\begin{frame}[t, fragile, shrink]
\frametitle{θ σύζευξη στην {\en SQL} 1:1}
\begin{minipage}{\wE}
\begin{exampleblock}{\small Να βρεθούν τα ονοματεπώνυμα των υπαλλήλων που είναι διευθυντές τμημάτων}
\en
\begin{SQL}
  SELECT firstname, lastname
    FROM departments, employees
   WHERE departments.manager = employees.empid;
\end{SQL}
\el
\end{exampleblock}
\pause
\begin{exampleblock}{\small Να βρεθεί το ονοματεπώνυμο του υπαλλήλου που διευθύνει το τμήμα 2}
\en
\begin{SQL}
  SELECT firstname, lastname
    FROM departments, employees
   WHERE departments.manager = employees.empid
     AND departments.depid = 2;
\end{SQL}
\el
\end{exampleblock}
\end{minipage}
\end{frame}


\begin{frame}[t, fragile, shrink]
\frametitle{θ σύζευξη στην {\en SQL} Ν:Ν}
\begin{minipage}{\wE}
\begin{exampleblock}{\small Να βρεθεί ο τίτλος των έργων και τα ονοματεπώνυμα των υπαλλήλων που συμμετέχουν σε αυτά.}
\en
\begin{SQL}
  SELECT title, firstname, lastname
    FROM employees, workson, projects
   WHERE employees.empid = workson.empid
     AND workson.proid = projects.proid;
\end{SQL}
\el
\end{exampleblock}
\pause
\begin{alertblock}{\small Να βρεθεί ο τίτλος των έργων και τα ονοματεπώνυμα των υπαλλήλων που συμμετέχουν σε αυτά.}
\en
\begin{SQL}
  SELECT title, firstname, lastname
    FROM employees, projects
   WHERE employees.empid = workson.empid
     AND workson.proid = projects.proid;
\end{SQL}
\el
\end{alertblock}
\end{minipage}
\end{frame}


\begin{frame}[t, fragile, shrink]
\frametitle{θ ανισοσύζευξη στην {\en SQL}}
\begin{minipage}{\wE}
\vspace*{-0.5cm}
\begin{alertblock}{\small Να βρεθούν τα ονοματεπώνυμα των υπαλλήλων που προσλήφθηκαν μετά την έναρξη του έργου με κωδικό 21.}
\en
\begin{SQL}
  SELECT firstname, lastname
    FROM employees, workson, projects
   WHERE employees.empid = workson.empid
     AND workson.proid = projects.proid
     AND employees.hiredate > projects.startdate
     AND projects.proid = 21;
\end{SQL}
\el
\end{alertblock}
\pause
\vspace*{-0.4cm}
\begin{exampleblock}{\small Να βρεθούν τα ονοματεπώνυμα των υπαλλήλων που προσλήφθηκαν μετά την έναρξη του έργου με κωδικό 21.}
\en
\begin{SQL}
  SELECT firstname, lastname
    FROM employees, projects
   WHERE employees.hiredate > projects.startdate
     AND projects.proid = 21;
\end{SQL}
\el
\end{exampleblock}
\end{minipage}
\end{frame}



\begin{frame}[t, fragile, shrink]
\frametitle{Μετονομασία πινάκων}
\begin{minipage}{\wE}
\vspace*{-0.5cm}
\begin{block}{Με χρήση του {\en AS}}
\en
\begin{SQL}
    FROM employees AS e
\end{SQL}
\el
\end{block}
\vspace*{-0.4cm}
\begin{block}{Χωρίς χρήση του {\en AS}}
\en
\begin{SQL}
    FROM employees e
\end{SQL}
\el
\end{block}
\pause
\vspace*{-0.4cm}
\begin{block}{Στη σύζευξη}
\en
\begin{SQL}
    FROM departments d, employees e
\end{SQL}
\el
\end{block}
\pause
\vspace*{-0.4cm}
\begin{block}{Στο ερώτημα}
\en
\begin{SQL}
  SELECT e.firstname, e.lastname
    FROM employees e, projects p
   WHERE e.hiredate > p.startdate
     AND p.proid = 21;   
\end{SQL}
\el
\end{block}
\end{minipage}
\end{frame}
