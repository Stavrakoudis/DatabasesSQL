\section[]{\textgreek{Εσωτερική σύζευξη πινάκων στην \textlatin{SQL}}}


\begin{frame}[t, fragile, shrink]
\frametitle{Εσωτερική σύζευξη στην {\en SQL}}
\begin{minipage}{0.92\textwidth}
\begin{exampleblock}{Σύζευξη τμημάτων και υπαλλήλων}
\en
\begin{SQL}
  SELECT *
    FROM departments INNER JOIN employees
         ON departments.depid = employees.depid;
\end{SQL}
\el
\end{exampleblock}
\pause
\begin{enumerate}[<+->] \pause \itemsep 6pt
  \item Οι δύο πίνακες ενώνονται με τον όρο {\sq INNER JOIN}.
  \item Αντί για {\sq WHERE} υπάρχει ({\bb \el υποχρεωτικά}) μετά το {\sq INNER JOIN}
        η φράση {\sq ON}.
  \item Το πεδίο σύζευξης (εδώ το {\ra depid}) υπάρχει δύο φορές στο αποτέλεσμα.  
  \item Το πεδίο σύζευξης μπορεί να έχει διαφορετικό όνομα στους δύο πίνακες.       
\end{enumerate}
\end{minipage}
\end{frame}


\begin{frame}[t, fragile, shrink]
\frametitle{Ισοδυναμία εσωτερικής και θ σύζευξης στην {\en SQL}}
\begin{minipage}{0.92\textwidth}
\begin{block}{\small Σύζευξη τμημάτων και υπαλλήλων με εσωτερική σύζευξη}
\en
\begin{SQL}
  SELECT *
    FROM departments INNER JOIN employees
         ON departments.depid = employees.depid;
\end{SQL}
\el
\end{block}
\begin{block}{\small Σύζευξη τμημάτων και υπαλλήλων με θ σύζευξη}
\en
\begin{SQL}
  SELECT *
    FROM departments JOIN employees
   WHERE departments.depid = employees.depid;
\end{SQL}
\el
\end{block}
\end{minipage}
\end{frame}

\begin{frame}[t, fragile, shrink]
\frametitle{Όχι στις υπερβολές}
\begin{minipage}{\wE}
\begin{alertblock}{\small Σύζευξη τμημάτων και υπαλλήλων με εσωτερική και θ σύζευξη}
\en
\begin{SQL}
  SELECT *
    FROM departments INNER JOIN employees
         ON departments.depid = employees.depid
   WHERE departments.depid = employees.depid;
\end{SQL}
\el
\end{alertblock}
\pause
\begin{enumerate} \itemsep 6pt
  \item Περιττό.
  \item Δεν είναι συντακτικά λάθος, είναι όμως εννοιολογικά μπερδεμένο.
  \item Μία φορά αρκεί, η πολυλογία φέρνει λάθη. 
  \item Δυσκολία συντήρησης του κώδικα.
\end{enumerate}
\end{minipage}
\end{frame}



\begin{frame}[t, fragile, shrink]
\frametitle{Μετονομασία πινάκων}
\begin{minipage}{\wE}
\vspace*{-0.5cm}
\begin{block}{Με χρήση του {\en AS}}
\en
\begin{SQL}
    FROM employees AS e
\end{SQL}
\el
\end{block}
\vspace*{-0.4cm}
\begin{block}{Χωρίς χρήση του {\en AS}}
\en
\begin{SQL}
    FROM employees e
\end{SQL}
\el
\end{block}
\pause
\vspace*{-0.4cm}
\begin{block}{Στη σύζευξη}
\en
\begin{SQL}
    FROM departments d INNER JOIN employees e
\end{SQL}
\el
\end{block}
\pause
\begin{block}{Στο ερώτημα}
\en
\begin{SQL}
  SELECT *
    FROM departments d INNER JOIN employees e;   
\end{SQL}
\el
\end{block}
\end{minipage}
\end{frame}



\begin{frame}[t, fragile, shrink]
\frametitle{Παράδειγμα  {\en INNER JOIN} -- 1}
\vspace*{-0.5cm}
\begin{minipage}{\wE}
\begin{exampleblock}{\small Να δοθεί το επώνυμο των υπαλλήλων και το όνομα του τμήματος στο
              οποίο εργάζονται}
\en
\[
\begin{split}
  \Pi_{firstname, lastname, depname} \\ ( departments \bowtie_{departments.depid=employees.depid} employees)
\end{split}
\]
\pause
\vspace*{-0.5cm}
\en
\begin{SQL}
  SELECT employees.lastname, departments.depname
    FROM departments INNER JOIN employees
         ON employees.depid = departments.depid;
\end{SQL}
\end{exampleblock}
\el
\begin{itemize}
  \item Η σύζευξη γίνεται με βάση το κοινό του πεδίο {\ra depid}.
  \item Η σύζευξη με βάση {\crr πρωτεύον} και {\crr ξένο} κλειδί  
        είναι η πλέον συνηθισμένη περίπτωση σύζευξης.
\end{itemize}
\end{minipage}
\end{frame}




\begin{frame}[t, fragile, shrink]
\frametitle{Παράδειγμα  {\en INNER JOIN} -- 2}
\begin{minipage}{\wE}
\vspace*{-0.5cm}
\begin{exampleblock}{\small Να δοθεί το όνομα των υπαλλήλων, το όνομα του τμήματος στο
              οποίο εργάζονται, και ο μισθός τους}
\en
\[
\begin{array}{l}
  \Pi_{firstname, lastname, depname, salary}  \\
  ( \varrho_{d} (departments) \bowtie_{d.depid=e.depid} \varrho_{e} (employees) )
\end{array}
\]
\pause
\vspace*{-0.5cm}
\en
\begin{SQL}
  SELECT e.firstname, e.lastname, d.depname, e.salary
    FROM employees e INNER JOIN departments d
         ON e.depid = d.depid;
\end{SQL}
\end{exampleblock}
\el
\footnotesize
\begin{tabular}{l l l r} \hline
{\bb\en firstname} & {\bb\en lastname} & {\bb\en depname} & {\bb\en salary} \\ \hline
Μαρία & Αθανασίου & Διοίκησης/Επίβλεψης & 2787.69 \\
Κρινιώ & Μαροπούλου & Διοίκησης/Επίβλεψης & 1754.67 \\
Κυριάκος & Ρούσσης & Διοίκησης/Επίβλεψης & 1852.99 \\
Μαρία & Αλεβιζάτου & Οικονομoλόγων/Λογιστών & 1321.92 \\
Δέσποινα & Παπαδοπούλου & Οικονομoλόγων/Λογιστών & 1609.52 \\
Πέτρος & Αρβανιτάκης & Οικονομoλόγων/Λογιστών & 1323.80 \\
 $\ldots$ & $\ldots$ & $\ldots$ & $\ldots$  \\  \hline
\end{tabular}
\end{minipage}
\end{frame}


\begin{frame}[t, fragile, shrink]
\frametitle{Παράδειγμα  {\en INNER JOIN} -- 3}
\begin{minipage}{\wE}
\vspace*{-0.5cm}
\begin{exampleblock}{\small Να δοθεί το όνομα των υπαλλήλων, το όνομα του τμήματος στο
              οποίο εργάζονται και ο μισθός τους, για υπαλλήλους με μισθό
              μεταξύ 1050 και 1300 \euro}
\en
\[
\begin{array}{l}
  \Pi_{firstname, lastname, depname, salary}
  (\sigma_{salary\geq1050 \wedge salary\leq1300} \\
  ( \varrho_{d} (departments) \bowtie_{d.depid=e.depid} \varrho_{e} (employees) ) )
\end{array}
\]
\pause
\vspace*{-0.5cm}
\en
\begin{SQL}
  SELECT e.firstname, e.lastname, d.depname, e.salary
    FROM employees e INNER JOIN departments d
         ON e.depid = d.depid
   WHERE e.salary BETWEEN 1050 AND 1300;
\end{SQL}
\end{exampleblock}
\el
\begin{itemize}
  \item Ο όρος {\sq WHERE} μπορεί να χρησιμοποιηθεί  για περιορισμό  των εγγραφών.
  \item Η παράσταση συνθήκης μπορεί να αφορά οποιοδήποτε   πεδίο
        από αυτά που υπάρχουν στους δύο πίνακες.
\end{itemize}
\end{minipage}
\end{frame}


\begin{frame}[t, fragile, shrink]
\frametitle{Παράδειγμα  {\en INNER JOIN} -- 4}

\vspace*{-0.5cm}
\begin{exampleblock}{\small Κωδικός και όνομα όλων των υπαλλήλων
             που συμμετέχουν στο έργο με κωδικό 38,
             με αύξουσα ταξινόμηση ως προς το επώνυμο}
\en
\vspace*{-0.5cm}
\[
  \Pi_{empid, firstname, lastname}
  (\sigma_{proid = 38}
    ( \varrho_{e} (employees) \bowtie_{e.empid=w.empid} \varrho_{w} (workson) )
  )
\]
\pause
\vspace*{-0.6cm}
\en
\begin{SQL}
  SELECT e.empid, e.firstname, e.lastname
    FROM employees e INNER JOIN workson w
         ON e.empid = w.empid
   WHERE w.proid = 38
ORDER BY e.lastname ASC;
\end{SQL}
\end{exampleblock}
\el
\begin{minipage}{\wE}
\begin{itemize}
  \item Ο όρος {\sq ORDER BY} πάντα στο τέλος.
  \item Προσέξτε πως χρειάζεται {\bb σύζευξη} ακόμα και αν όλα τα πεδία
        που εμφανίζονται μετά τον όρο {\sq SELECT} βρίσκονται σε ένα πίνακα.
  \item Ξαναγράψτε την απάντηση με υποερώτημα (παρακάτω μάθημα).      
\end{itemize}
\end{minipage}
\end{frame}


\begin{frame}[t, fragile, shrink]
\frametitle{Πολλά προς πολλά}
\begin{minipage}{\wE}
\begin{exampleblock}{\small Να βρεθούν τα ονόματα των υπαλλήλων και ο κωδικός και ο προϋπολογισμός των
              έργων στα οποία συμμετέχουν  υπάλληλοι με μισθό μεγαλύτερο από \euro1700 }
\en
\vspace{-0.5cm}
\[
\begin{array}{l}
  \Pi_{firstname, lastname, proid, budget}
  (\sigma_{salary>1700}                    \\
    ( \varrho_{e} (employees) \bowtie_{d.depid=e.depid} \varrho_{w} (workson)
      \bowtie_{w.proid=p.proid} \varrho_{p} (projects)   )
  )
\end{array}
\]
\pause
\vspace{-0.5cm}
\en
\begin{SQL}
  SELECT e.firstname, e.lastname, p.proid, p.budget
    FROM (employees e INNER JOIN workson w
         ON e.empid = w.empid)
                      INNER JOIN projects p
         ON p.proid = w.proid
   WHERE e.salary > 1700;
\end{SQL}
\el
\end{exampleblock}
\end{minipage}
\end{frame}

\begin{frame}[t, fragile, shrink]
\frametitle{Πολλά προς πολλά -- Εναλλακτικός τρόπος}
\begin{minipage}{\wE}
\begin{exampleblock}{\small Να βρεθούν τα ονόματα των υπαλλήλων και ο κωδικός και ο προϋπολογισμός των
              έργων στα οποία συμμετέχουν  υπάλληλοι με μισθό μεγαλύτερο του 1700\euro}
\en
\[
\begin{array}{l}
  \Pi_{firstname, lastname, proid, budget}
  (\sigma_{e.empid = w.empid \wedge p.proid = w.proid \wedge e.salary>1700 }     \\
    ( \varrho_{e} (employees) \times \varrho_{w} (workson)
      \times \varrho_{p} (projects)   )
  )
\end{array}
\]
\begin{SQL}
  SELECT e.firstname, e.lastname, p.proid, p.budget
    FROM employees e, workson w, projects p
   WHERE e.empid = w.empid
     AND p.proid = w.proid
     AND e.salary > 1700;
\end{SQL}
\el
\end{exampleblock}
\end{minipage}
\end{frame}


\begin{frame}[t, fragile, shrink]
\frametitle{Ερώτημα με 4 πίνακες}
\begin{minipage}{\wE}
\vspace{-0.5cm}
\begin{exampleblock}{\footnotesize Να βρεθεί το όνομα των υπαλλήλων και του τμήματος των υπαλλήλων
             για όλους τους υπαλλήλους που προσλήφθηκαν μετά από την 1/1/2002 και
             απασχολούνται σε έργα με βαθμό προόδου τουλάχιστον 20\%}
\vspace{-0.5cm}             
\en
\[
\begin{array}{l}
 \Pi_{firstname, lastname, depname}
 (\sigma_{hiredate>'2002-01-01' \wedge budget>100000} \\
 ( \varrho_{d} (departments) \bowtie_{d.depid=e.depid}
    \varrho_{e} (employees)  \\ \bowtie_{e.empid=w.empid} \varrho_{w} (workson)
     \bowtie_{w.proid=p.proid} \varrho_{p} (projects)   )
\end{array}
\]
\pause
\vspace{-0.5cm}
\en
\begin{SQL}
  SELECT DISTINCT e.firstname, e.lastname, d.depname
    FROM ((departments d INNER JOIN employees e
         ON d.depid = e.depid)
                         INNER JOIN workson w
         ON e.empid = w.empid)
                         INNER JOIN projects p
         ON p.proid = w.proid
   WHERE e.hiredate > '2002-01-01'
     AND p.progress > 20;
\end{SQL}
\el
\end{exampleblock}
\end{minipage}
\end{frame}

